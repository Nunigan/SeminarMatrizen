%
% part1.tex
%
% (c) 2018 Prof Dr Andreas Müller, Hochschule Rapperswil
%
\begin{refsection}
%
% vorwort.tex -- Vorwort zum Buch zum Seminar
%
% (c) 2019 Prof Dr Andreas Mueller, Hochschule Rapperswil
%
\chapter*{Vorwort}
\lhead{Vorwort}
\rhead{}
Dieses Buch entstand im Rahmen des Mathematischen Seminars
im Frühjahrssemester 2021 an der Ostschweizer Fachhochschule in Rapperswil.
Die Teilnehmer, Studierende der Studiengänge für Elektrotechnik,
Informatik und Bauingenieurwesen 
der OST, erarbeiteten nach einer Einführung in das Themengebiet jeweils
einzelne Aspekte des Gebietes in Form einer Seminararbeit, über
deren Resultate sie auch in einem Vortrag informierten. 

Im Frühjahr 2021 war das Thema des Seminars die Matrizen.
Ziel war, die Vielfalt der Anwendungsmöglichkeiten dieser einfachen
Datenstruktur zu zeigen.

In einigen Arbeiten wurde auch Code zur Demonstration der 
besprochenen Methoden und Resultate geschrieben, soweit
möglich und sinnvoll wurde dieser Code im Github-Repository
\index{Github-Repository}%
dieses Kurses%
\footnote{\url{https://github.com/AndreasFMueller/SeminarMatrizen.git}}
\cite{buch:repo}
abgelegt.
Im genannten Repository findet sich auch der Source-Code dieses
Skriptes, es wird hier unter einer Creative Commons Lizenz
zur Verfügung gestellt.


\part{Grundlagen}
%
% chapter.tex
%
% (c) 2021 Prof Dr Andreas Müller, Hochschule Rapperswi
%
\folie{5/plan.tex}
\folie{5/planbeispiele.tex}
\folie{5/verzerrung.tex}
\folie{5/motivation.tex}
\folie{5/charpoly.tex}
\folie{5/kernbildintro.tex}
\folie{5/kernbilder.tex}
\folie{5/kernbild.tex}
\folie{5/ketten.tex}
\folie{5/dimension.tex}
\folie{5/folgerungen.tex}
\folie{5/injektiv.tex}
\folie{5/nilpotent.tex}
\folie{5/eigenraeume.tex}
\folie{5/zerlegung.tex}
\folie{5/normalnilp.tex}
\folie{5/bloecke.tex}
\folie{5/jordanblock.tex}
\folie{5/jordan.tex}
\folie{5/reellenormalform.tex}
\folie{5/cayleyhamilton.tex}
\folie{5/konvergenzradius.tex}
\folie{5/krbeispiele.tex}
\folie{5/spektralgelfand.tex}
\folie{5/Aiteration.tex}
\folie{5/satzvongelfand.tex}
\folie{5/stoneweierstrass.tex}
\folie{5/swbeweis.tex}
\folie{5/potenzreihenmethode.tex}
\folie{5/logarithmusreihe.tex}
\folie{5/exponentialfunktion.tex}
\folie{5/hyperbolisch.tex}
\folie{5/spektrum.tex}
\folie{5/normal.tex}
\folie{5/normalbeispiel.tex}
\folie{5/normalbeispiel34.tex}
\folie{5/approximation.tex}

%
% chapter.tex
%
% (c) 2021 Prof Dr Andreas Müller, Hochschule Rapperswi
%
\folie{5/plan.tex}
\folie{5/planbeispiele.tex}
\folie{5/verzerrung.tex}
\folie{5/motivation.tex}
\folie{5/charpoly.tex}
\folie{5/kernbildintro.tex}
\folie{5/kernbilder.tex}
\folie{5/kernbild.tex}
\folie{5/ketten.tex}
\folie{5/dimension.tex}
\folie{5/folgerungen.tex}
\folie{5/injektiv.tex}
\folie{5/nilpotent.tex}
\folie{5/eigenraeume.tex}
\folie{5/zerlegung.tex}
\folie{5/normalnilp.tex}
\folie{5/bloecke.tex}
\folie{5/jordanblock.tex}
\folie{5/jordan.tex}
\folie{5/reellenormalform.tex}
\folie{5/cayleyhamilton.tex}
\folie{5/konvergenzradius.tex}
\folie{5/krbeispiele.tex}
\folie{5/spektralgelfand.tex}
\folie{5/Aiteration.tex}
\folie{5/satzvongelfand.tex}
\folie{5/stoneweierstrass.tex}
\folie{5/swbeweis.tex}
\folie{5/potenzreihenmethode.tex}
\folie{5/logarithmusreihe.tex}
\folie{5/exponentialfunktion.tex}
\folie{5/hyperbolisch.tex}
\folie{5/spektrum.tex}
\folie{5/normal.tex}
\folie{5/normalbeispiel.tex}
\folie{5/normalbeispiel34.tex}
\folie{5/approximation.tex}

%
% chapter.tex
%
% (c) 2021 Prof Dr Andreas Müller, Hochschule Rapperswi
%
\folie{5/plan.tex}
\folie{5/planbeispiele.tex}
\folie{5/verzerrung.tex}
\folie{5/motivation.tex}
\folie{5/charpoly.tex}
\folie{5/kernbildintro.tex}
\folie{5/kernbilder.tex}
\folie{5/kernbild.tex}
\folie{5/ketten.tex}
\folie{5/dimension.tex}
\folie{5/folgerungen.tex}
\folie{5/injektiv.tex}
\folie{5/nilpotent.tex}
\folie{5/eigenraeume.tex}
\folie{5/zerlegung.tex}
\folie{5/normalnilp.tex}
\folie{5/bloecke.tex}
\folie{5/jordanblock.tex}
\folie{5/jordan.tex}
\folie{5/reellenormalform.tex}
\folie{5/cayleyhamilton.tex}
\folie{5/konvergenzradius.tex}
\folie{5/krbeispiele.tex}
\folie{5/spektralgelfand.tex}
\folie{5/Aiteration.tex}
\folie{5/satzvongelfand.tex}
\folie{5/stoneweierstrass.tex}
\folie{5/swbeweis.tex}
\folie{5/potenzreihenmethode.tex}
\folie{5/logarithmusreihe.tex}
\folie{5/exponentialfunktion.tex}
\folie{5/hyperbolisch.tex}
\folie{5/spektrum.tex}
\folie{5/normal.tex}
\folie{5/normalbeispiel.tex}
\folie{5/normalbeispiel34.tex}
\folie{5/approximation.tex}

%
% chapter.tex
%
% (c) 2021 Prof Dr Andreas Müller, Hochschule Rapperswi
%
\folie{5/plan.tex}
\folie{5/planbeispiele.tex}
\folie{5/verzerrung.tex}
\folie{5/motivation.tex}
\folie{5/charpoly.tex}
\folie{5/kernbildintro.tex}
\folie{5/kernbilder.tex}
\folie{5/kernbild.tex}
\folie{5/ketten.tex}
\folie{5/dimension.tex}
\folie{5/folgerungen.tex}
\folie{5/injektiv.tex}
\folie{5/nilpotent.tex}
\folie{5/eigenraeume.tex}
\folie{5/zerlegung.tex}
\folie{5/normalnilp.tex}
\folie{5/bloecke.tex}
\folie{5/jordanblock.tex}
\folie{5/jordan.tex}
\folie{5/reellenormalform.tex}
\folie{5/cayleyhamilton.tex}
\folie{5/konvergenzradius.tex}
\folie{5/krbeispiele.tex}
\folie{5/spektralgelfand.tex}
\folie{5/Aiteration.tex}
\folie{5/satzvongelfand.tex}
\folie{5/stoneweierstrass.tex}
\folie{5/swbeweis.tex}
\folie{5/potenzreihenmethode.tex}
\folie{5/logarithmusreihe.tex}
\folie{5/exponentialfunktion.tex}
\folie{5/hyperbolisch.tex}
\folie{5/spektrum.tex}
\folie{5/normal.tex}
\folie{5/normalbeispiel.tex}
\folie{5/normalbeispiel34.tex}
\folie{5/approximation.tex}

%
% chapter.tex
%
% (c) 2021 Prof Dr Andreas Müller, Hochschule Rapperswi
%
\folie{5/plan.tex}
\folie{5/planbeispiele.tex}
\folie{5/verzerrung.tex}
\folie{5/motivation.tex}
\folie{5/charpoly.tex}
\folie{5/kernbildintro.tex}
\folie{5/kernbilder.tex}
\folie{5/kernbild.tex}
\folie{5/ketten.tex}
\folie{5/dimension.tex}
\folie{5/folgerungen.tex}
\folie{5/injektiv.tex}
\folie{5/nilpotent.tex}
\folie{5/eigenraeume.tex}
\folie{5/zerlegung.tex}
\folie{5/normalnilp.tex}
\folie{5/bloecke.tex}
\folie{5/jordanblock.tex}
\folie{5/jordan.tex}
\folie{5/reellenormalform.tex}
\folie{5/cayleyhamilton.tex}
\folie{5/konvergenzradius.tex}
\folie{5/krbeispiele.tex}
\folie{5/spektralgelfand.tex}
\folie{5/Aiteration.tex}
\folie{5/satzvongelfand.tex}
\folie{5/stoneweierstrass.tex}
\folie{5/swbeweis.tex}
\folie{5/potenzreihenmethode.tex}
\folie{5/logarithmusreihe.tex}
\folie{5/exponentialfunktion.tex}
\folie{5/hyperbolisch.tex}
\folie{5/spektrum.tex}
\folie{5/normal.tex}
\folie{5/normalbeispiel.tex}
\folie{5/normalbeispiel34.tex}
\folie{5/approximation.tex}

%
% chapter.tex
%
% (c) 2021 Prof Dr Andreas Müller, Hochschule Rapperswi
%
\folie{5/plan.tex}
\folie{5/planbeispiele.tex}
\folie{5/verzerrung.tex}
\folie{5/motivation.tex}
\folie{5/charpoly.tex}
\folie{5/kernbildintro.tex}
\folie{5/kernbilder.tex}
\folie{5/kernbild.tex}
\folie{5/ketten.tex}
\folie{5/dimension.tex}
\folie{5/folgerungen.tex}
\folie{5/injektiv.tex}
\folie{5/nilpotent.tex}
\folie{5/eigenraeume.tex}
\folie{5/zerlegung.tex}
\folie{5/normalnilp.tex}
\folie{5/bloecke.tex}
\folie{5/jordanblock.tex}
\folie{5/jordan.tex}
\folie{5/reellenormalform.tex}
\folie{5/cayleyhamilton.tex}
\folie{5/konvergenzradius.tex}
\folie{5/krbeispiele.tex}
\folie{5/spektralgelfand.tex}
\folie{5/Aiteration.tex}
\folie{5/satzvongelfand.tex}
\folie{5/stoneweierstrass.tex}
\folie{5/swbeweis.tex}
\folie{5/potenzreihenmethode.tex}
\folie{5/logarithmusreihe.tex}
\folie{5/exponentialfunktion.tex}
\folie{5/hyperbolisch.tex}
\folie{5/spektrum.tex}
\folie{5/normal.tex}
\folie{5/normalbeispiel.tex}
\folie{5/normalbeispiel34.tex}
\folie{5/approximation.tex}

%
% chapter.tex
%
% (c) 2021 Prof Dr Andreas Müller, Hochschule Rapperswi
%
\folie{5/plan.tex}
\folie{5/planbeispiele.tex}
\folie{5/verzerrung.tex}
\folie{5/motivation.tex}
\folie{5/charpoly.tex}
\folie{5/kernbildintro.tex}
\folie{5/kernbilder.tex}
\folie{5/kernbild.tex}
\folie{5/ketten.tex}
\folie{5/dimension.tex}
\folie{5/folgerungen.tex}
\folie{5/injektiv.tex}
\folie{5/nilpotent.tex}
\folie{5/eigenraeume.tex}
\folie{5/zerlegung.tex}
\folie{5/normalnilp.tex}
\folie{5/bloecke.tex}
\folie{5/jordanblock.tex}
\folie{5/jordan.tex}
\folie{5/reellenormalform.tex}
\folie{5/cayleyhamilton.tex}
\folie{5/konvergenzradius.tex}
\folie{5/krbeispiele.tex}
\folie{5/spektralgelfand.tex}
\folie{5/Aiteration.tex}
\folie{5/satzvongelfand.tex}
\folie{5/stoneweierstrass.tex}
\folie{5/swbeweis.tex}
\folie{5/potenzreihenmethode.tex}
\folie{5/logarithmusreihe.tex}
\folie{5/exponentialfunktion.tex}
\folie{5/hyperbolisch.tex}
\folie{5/spektrum.tex}
\folie{5/normal.tex}
\folie{5/normalbeispiel.tex}
\folie{5/normalbeispiel34.tex}
\folie{5/approximation.tex}

%
% chapter.tex
%
% (c) 2021 Prof Dr Andreas Müller, Hochschule Rapperswi
%
\folie{5/plan.tex}
\folie{5/planbeispiele.tex}
\folie{5/verzerrung.tex}
\folie{5/motivation.tex}
\folie{5/charpoly.tex}
\folie{5/kernbildintro.tex}
\folie{5/kernbilder.tex}
\folie{5/kernbild.tex}
\folie{5/ketten.tex}
\folie{5/dimension.tex}
\folie{5/folgerungen.tex}
\folie{5/injektiv.tex}
\folie{5/nilpotent.tex}
\folie{5/eigenraeume.tex}
\folie{5/zerlegung.tex}
\folie{5/normalnilp.tex}
\folie{5/bloecke.tex}
\folie{5/jordanblock.tex}
\folie{5/jordan.tex}
\folie{5/reellenormalform.tex}
\folie{5/cayleyhamilton.tex}
\folie{5/konvergenzradius.tex}
\folie{5/krbeispiele.tex}
\folie{5/spektralgelfand.tex}
\folie{5/Aiteration.tex}
\folie{5/satzvongelfand.tex}
\folie{5/stoneweierstrass.tex}
\folie{5/swbeweis.tex}
\folie{5/potenzreihenmethode.tex}
\folie{5/logarithmusreihe.tex}
\folie{5/exponentialfunktion.tex}
\folie{5/hyperbolisch.tex}
\folie{5/spektrum.tex}
\folie{5/normal.tex}
\folie{5/normalbeispiel.tex}
\folie{5/normalbeispiel34.tex}
\folie{5/approximation.tex}

%
% chapter.tex
%
% (c) 2021 Prof Dr Andreas Müller, Hochschule Rapperswi
%
\folie{5/plan.tex}
\folie{5/planbeispiele.tex}
\folie{5/verzerrung.tex}
\folie{5/motivation.tex}
\folie{5/charpoly.tex}
\folie{5/kernbildintro.tex}
\folie{5/kernbilder.tex}
\folie{5/kernbild.tex}
\folie{5/ketten.tex}
\folie{5/dimension.tex}
\folie{5/folgerungen.tex}
\folie{5/injektiv.tex}
\folie{5/nilpotent.tex}
\folie{5/eigenraeume.tex}
\folie{5/zerlegung.tex}
\folie{5/normalnilp.tex}
\folie{5/bloecke.tex}
\folie{5/jordanblock.tex}
\folie{5/jordan.tex}
\folie{5/reellenormalform.tex}
\folie{5/cayleyhamilton.tex}
\folie{5/konvergenzradius.tex}
\folie{5/krbeispiele.tex}
\folie{5/spektralgelfand.tex}
\folie{5/Aiteration.tex}
\folie{5/satzvongelfand.tex}
\folie{5/stoneweierstrass.tex}
\folie{5/swbeweis.tex}
\folie{5/potenzreihenmethode.tex}
\folie{5/logarithmusreihe.tex}
\folie{5/exponentialfunktion.tex}
\folie{5/hyperbolisch.tex}
\folie{5/spektrum.tex}
\folie{5/normal.tex}
\folie{5/normalbeispiel.tex}
\folie{5/normalbeispiel34.tex}
\folie{5/approximation.tex}

%
% chapter.tex
%
% (c) 2021 Prof Dr Andreas Müller, Hochschule Rapperswi
%
\folie{5/plan.tex}
\folie{5/planbeispiele.tex}
\folie{5/verzerrung.tex}
\folie{5/motivation.tex}
\folie{5/charpoly.tex}
\folie{5/kernbildintro.tex}
\folie{5/kernbilder.tex}
\folie{5/kernbild.tex}
\folie{5/ketten.tex}
\folie{5/dimension.tex}
\folie{5/folgerungen.tex}
\folie{5/injektiv.tex}
\folie{5/nilpotent.tex}
\folie{5/eigenraeume.tex}
\folie{5/zerlegung.tex}
\folie{5/normalnilp.tex}
\folie{5/bloecke.tex}
\folie{5/jordanblock.tex}
\folie{5/jordan.tex}
\folie{5/reellenormalform.tex}
\folie{5/cayleyhamilton.tex}
\folie{5/konvergenzradius.tex}
\folie{5/krbeispiele.tex}
\folie{5/spektralgelfand.tex}
\folie{5/Aiteration.tex}
\folie{5/satzvongelfand.tex}
\folie{5/stoneweierstrass.tex}
\folie{5/swbeweis.tex}
\folie{5/potenzreihenmethode.tex}
\folie{5/logarithmusreihe.tex}
\folie{5/exponentialfunktion.tex}
\folie{5/hyperbolisch.tex}
\folie{5/spektrum.tex}
\folie{5/normal.tex}
\folie{5/normalbeispiel.tex}
\folie{5/normalbeispiel34.tex}
\folie{5/approximation.tex}

%
% chapter.tex
%
% (c) 2021 Prof Dr Andreas Müller, Hochschule Rapperswi
%
\folie{5/plan.tex}
\folie{5/planbeispiele.tex}
\folie{5/verzerrung.tex}
\folie{5/motivation.tex}
\folie{5/charpoly.tex}
\folie{5/kernbildintro.tex}
\folie{5/kernbilder.tex}
\folie{5/kernbild.tex}
\folie{5/ketten.tex}
\folie{5/dimension.tex}
\folie{5/folgerungen.tex}
\folie{5/injektiv.tex}
\folie{5/nilpotent.tex}
\folie{5/eigenraeume.tex}
\folie{5/zerlegung.tex}
\folie{5/normalnilp.tex}
\folie{5/bloecke.tex}
\folie{5/jordanblock.tex}
\folie{5/jordan.tex}
\folie{5/reellenormalform.tex}
\folie{5/cayleyhamilton.tex}
\folie{5/konvergenzradius.tex}
\folie{5/krbeispiele.tex}
\folie{5/spektralgelfand.tex}
\folie{5/Aiteration.tex}
\folie{5/satzvongelfand.tex}
\folie{5/stoneweierstrass.tex}
\folie{5/swbeweis.tex}
\folie{5/potenzreihenmethode.tex}
\folie{5/logarithmusreihe.tex}
\folie{5/exponentialfunktion.tex}
\folie{5/hyperbolisch.tex}
\folie{5/spektrum.tex}
\folie{5/normal.tex}
\folie{5/normalbeispiel.tex}
\folie{5/normalbeispiel34.tex}
\folie{5/approximation.tex}

%
% chapter.tex
%
% (c) 2021 Prof Dr Andreas Müller, Hochschule Rapperswi
%
\folie{5/plan.tex}
\folie{5/planbeispiele.tex}
\folie{5/verzerrung.tex}
\folie{5/motivation.tex}
\folie{5/charpoly.tex}
\folie{5/kernbildintro.tex}
\folie{5/kernbilder.tex}
\folie{5/kernbild.tex}
\folie{5/ketten.tex}
\folie{5/dimension.tex}
\folie{5/folgerungen.tex}
\folie{5/injektiv.tex}
\folie{5/nilpotent.tex}
\folie{5/eigenraeume.tex}
\folie{5/zerlegung.tex}
\folie{5/normalnilp.tex}
\folie{5/bloecke.tex}
\folie{5/jordanblock.tex}
\folie{5/jordan.tex}
\folie{5/reellenormalform.tex}
\folie{5/cayleyhamilton.tex}
\folie{5/konvergenzradius.tex}
\folie{5/krbeispiele.tex}
\folie{5/spektralgelfand.tex}
\folie{5/Aiteration.tex}
\folie{5/satzvongelfand.tex}
\folie{5/stoneweierstrass.tex}
\folie{5/swbeweis.tex}
\folie{5/potenzreihenmethode.tex}
\folie{5/logarithmusreihe.tex}
\folie{5/exponentialfunktion.tex}
\folie{5/hyperbolisch.tex}
\folie{5/spektrum.tex}
\folie{5/normal.tex}
\folie{5/normalbeispiel.tex}
\folie{5/normalbeispiel34.tex}
\folie{5/approximation.tex}

%\begin{appendices}
%\end{appendices}
\vfill
\pagebreak
\ifodd\value{page}\else\null\clearpage\fi
\lhead{Literatur}
\rhead{}
\printbibliography[heading=subbibliography]
\label{buch:literatur}
\end{refsection}


