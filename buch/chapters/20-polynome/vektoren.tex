%
% vektoren.tex -- Darstellung von Polynomen als Vektoren
%
% (c) 2021 Prof Dr Andreas Müller, OST Ostschweizer Fachhochschule Rapperswil
%
\section{Polynome als Vektoren
\label{buch:section:polynome:vektoren}}
\rhead{Polynome als Vektoren}
Ein Polynom
\[
p(X) = a_nX^n + a_{n-1}X^{n-1} + \dots a_1X+a_0
\]
mit Koeffizienten in einem Ring $R$
ist spezifiziert, wenn die Koeffizienten $a_k$ bekannt sind.
Die Potenzen von $X$ dienen hier nur dazu, die verschiedenen
Koeffizienten zu unterscheiden.
Das Polynom $p(X)$ vom Grad $n$ ist also auch gegeben durch den
$n+1$-dimensionalen Vektor
\[
\begin{pmatrix}
a_0\\
a_1\\
\vdots\\
a_{n-1}\\
a_{n}
\end{pmatrix}
\in
R^n.
\]
Diese Darstellung eines Polynoms gibt auch die Addition von Polynomen
und die Multiplikation von Polynomen mit Skalaren aus $R$ korrekt wieder.
Die Abbildung von Vektoren auf Polynome
\[
\varphi
\colon  R^n \to R[X]
:
\begin{pmatrix}a_0\\\vdots\\a_n\end{pmatrix}
\mapsto
a_nX^n + a_{n-1}X^{n-1}+\dots+a_1X+a_0
\]
erfüllt also
\[
\varphi( \lambda a) = \lambda \varphi(a)
\qquad\text{und}\qquad
\varphi(a+b) = \varphi(a) + \varphi(b)
\]
und ist damit eine lineare Abbildung.
Umgekehrt kann man auch zu jedem Polynom $p(X)$ vom Grad~$\le n$ einen
Vektor finden, der von $\varphi$ auf das Polynom $p(X)$ abgebildet wird.
Die Abbildung $\varphi$ ist also ein Isomorphismus
\[
\varphi
\colon
\{p\in R[X]\;|\; \deg(p) \le n\}
\overset{\equiv}{\to}
R^{n+1}
\]
zwischen der Menge
der Polynome vom Grad $\le n$ auf $R^{n+1}$.
Für alle Rechnungen, bei denen es nur um Addition von Polynomen oder
um Multiplikation mit Skalaren geht, ist also diese vektorielle Darstellung
mit Hilfe von $\varphi$ eine zweckmässige Darstellung.

In zwei Bereichen ist die Beschreibung von Polynomen mit Vektoren allerdings
ungenügend: einerseits können Polynome können beliebig hohen Grad haben,
während Vektoren in $R^{n+1}$ höchstens $n+1$ Komponenten haben können.
Andererseits geht bei der vektoriellen Beschreibung die multiplikative
Struktur vollständig verloren.

\subsection{Polynome beliebigen Grades
\label{buch:subsection:polynome:beliebigergrad}}
Ein Polynom
\[
q(X)
=
b_mX^m + b_{m-1}X^{m-1} + \dots + b_1X + b_0
\]
vom Grad $m<n$ kann dargestellt werden als ein Vektor
\[
\begin{pmatrix}
b_0\\
b_1\\
\vdots\\
b_{m-1}\\
b_{m}\\
0\\
\vdots
\end{pmatrix}
\in
R^{n+1}
\]
mit der Eigenschaft, dass die Komponenten mit Indizes
$m+1,\dots n$ verschwinden.
Polynome vom Grad $m<n$ bilden einen Unterraum der Polynome vom Grad $n$.
Wir können auch die $m+1$-dimensionalen Vektoren in den $n+1$-dimensionalen
Vektoren einbetten, indem wir die Vektoren durch ``auffüllen'' mit Nullen
auf die richtige Länge bringen.
Es gibt also eine lineare Abbildung
\[
R^{m+1} \to R^{n+1}
\colon
\begin{pmatrix}
b_0\\b_1\\\vdots\\b_m
\end{pmatrix}
\mapsto
\begin{pmatrix}
b_0\\b_1\\\vdots\\b_m\\0\\\vdots
\end{pmatrix}
.
\]
Die Moduln $R^{k}$ sind also alle ineinandergeschachtelt, können aber
alle auf konsistente Weise mit der Abbildung $\varphi$ in den Polynomring
$R[X]$ abgebildet werden.
\begin{center}
\begin{tikzcd}
\{0\}\ar[r] %\arrow[d,"\varphi"]
	&R \ar[r] %\arrow[d, "\varphi"]
		&R^2 \ar[r] %\arrow[d, "\varphi"]
			&\dots \ar[r]
				&R^k \ar[r] %\arrow[d, "\varphi"]
					&R^{k+1} \ar[r] %\arrow[d, "\varphi"]
						&\dots
\\
R^{(0)}[X]\arrow[r,hook] \arrow[drrr,hook]
	&R^{(1)}[X]\arrow[r,hook] \arrow[drr,hook]
		&R^{(2)}[X]\arrow[r,hook] \arrow[dr,hook]
			&\dots\arrow[r,hook]
				&R^{(k)}[X]\arrow[r,hook] \arrow[dl,hook]
					&R^{(k+1)}[X]\arrow[r,hook] \arrow[dll,hook]
						&\dots
\\
	&
		&
			&R[X]
				&
					&
						&
\end{tikzcd}
\end{center}
\subsection{Multiplikative Struktur
\label{buch:subsection:polynome:multiplikativestruktur}}





