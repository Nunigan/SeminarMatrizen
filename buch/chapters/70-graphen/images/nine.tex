%
% nine.tex -- Nine node graph to illustrate Wilf's theorem
%
% (c) 2021 Prof Dr Andreas Müller, OST Ostschweizer Fachhochschule
%
\documentclass[tikz]{standalone}
\usepackage{amsmath}
\usepackage{times}
\usepackage{txfonts}
\usepackage{pgfplots}
\usepackage{csvsimple}
\usetikzlibrary{arrows,intersections,math}
\begin{document}
\def\skala{1}
\def\kante#1#2{
        \draw[shorten >= 0.2cm,shorten <= 0.2cm] (#1) -- (#2);
}
\def\knoten#1#2{
                \fill[color=#2!30] (#1) circle[radius=0.2];
                \draw[color=#2] (#1) circle[radius=0.2];
                \draw (#1) circle[radius=0.2];
}
\def\R{1.5}
\definecolor{rot}{rgb}{1,0,0}
\definecolor{gruen}{rgb}{0,0.6,0}
\definecolor{blau}{rgb}{0,0,1}

\begin{tikzpicture}[>=latex,thick,scale=\skala]

\coordinate (A) at (0:\R);
\coordinate (B) at (40:\R);
\coordinate (C) at (80:\R);
\coordinate (D) at (120:\R);
\coordinate (E) at (160:\R);
\coordinate (F) at (200:\R);
\coordinate (G) at (240:\R);
\coordinate (H) at (280:\R);
\coordinate (I) at (320:\R);

\knoten{A}{rot}
\knoten{B}{blau}
\knoten{C}{gruen}
\knoten{D}{blau}
\knoten{E}{rot}
\knoten{F}{blau}
\knoten{G}{rot}
\knoten{H}{gruen}
\knoten{I}{blau}

\kante{A}{B}
\kante{B}{C}
\kante{C}{D}
\kante{D}{E}
\kante{E}{F}
\kante{F}{G}
\kante{G}{H}
\kante{H}{I}
\kante{I}{A}

\kante{A}{C}
\kante{A}{D}
\kante{D}{G}


\end{tikzpicture}
\end{document}

