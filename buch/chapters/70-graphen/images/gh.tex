%
% gh.tex -- Lokalsierungsfunktionen für Wavelets auf einem Graphen
%
% (c) 2021 Prof Dr Andreas Müller, OST Ostschweizer Fachhochschule
%
\documentclass[tikz]{standalone}
\usepackage{amsmath}
\usepackage{times}
\usepackage{txfonts}
\usepackage{pgfplots}
\usepackage{csvsimple}
\usetikzlibrary{arrows,intersections,math}
\begin{document}
\def\skala{1}
\begin{tikzpicture}[>=latex,thick,scale=\skala]
\definecolor{darkgreen}{rgb}{0,0.6,0}

\def\kurve#1#2{
	\draw[color=#2,line width=1.4pt]
		plot[domain=0:6.3,samples=400]
			({\x},{7*\x*exp(-(\x/#1)*(\x/#1))/#1});
}

\begin{scope}

\draw[->] (-0.1,0) -- (6.6,0) coordinate[label={$\lambda$}];

\kurve{1}{red}
\foreach \k in {0,...,4}{
	\pgfmathparse{0.30*exp(ln(2)*\k)}
	\xdef\l{\pgfmathresult}
	\kurve{\l}{blue}
}

\node[color=red] at ({0.7*1},3) [above] {$g(\lambda)$};
\node[color=blue] at ({0.7*0.3*16},3) [above] {$g_i(\lambda)$};

\draw[->] (0,-0.1) -- (0,3.3);
\end{scope}

\begin{scope}[xshift=7cm]

\draw[->] (-0.1,0) -- (6.6,0) coordinate[label={$\lambda$}];

\draw[color=darkgreen,line width=1.4pt]
	plot[domain=0:6.3,samples=100] 
		({\x},{3*exp(-(\x/0.5)*(\x/0.5)});

\draw[->] (0,-0.1) -- (0,3.3) coordinate[label={right:$\color{darkgreen}h(\lambda)$}];

\end{scope}

\end{tikzpicture}
\end{document}

