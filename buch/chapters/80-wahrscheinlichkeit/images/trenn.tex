%
% trenn.tex -- Trenntrick graphische Darstellung
%
% (c) 2021 Prof Dr Andreas Müller, OST Ostschweizer Fachhochschule
%
\documentclass[tikz]{standalone}
\usepackage{amsmath}
\usepackage{times}
\usepackage{txfonts}
\usepackage{pgfplots}
\usepackage{csvsimple}
\usetikzlibrary{arrows,intersections,math}
\begin{document}
\def\skala{1}
\begin{tikzpicture}[>=latex,thick,scale=\skala]

\def\d{6}

\coordinate (u) at (5,3);
\coordinate (v) at (3,1);
\coordinate (ve) at (5,1.666);

\fill[color=gray!40] (0,0) rectangle (u);

\begin{scope}
\clip (0,0) rectangle (6.1,4.1);
\draw[color=red] (0,0) -- (9,3);
\end{scope}

\draw[->] (-0.1,0) -- (6.3,0) coordinate[label={$x_1$}];
\draw[->] (0,-0.1) -- (0,4.3) coordinate[label={right:$x_2$}];

\fill (u) circle[radius=0.05];
\node at (u) [above right] {$u$};

\fill (v) circle[radius=0.05];
\node at (v) [above right] {$v$};

\fill[color=red] (ve) circle[radius=0.05];
\node[color=red] at (ve) [above,rotate={atan(1/3)}] {$(1+\varepsilon)v$};

\end{tikzpicture}
\end{document}

