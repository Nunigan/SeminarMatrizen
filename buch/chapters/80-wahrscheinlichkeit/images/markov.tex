%
% markov2.tex -- template for standalon tikz images
%
% (c) 2021 Prof Dr Andreas Müller, OST Ostschweizer Fachhochschule
%
\documentclass[tikz]{standalone}
\usepackage{amsmath}
\usepackage{times}
\usepackage{txfonts}
\usepackage{pgfplots}
\usepackage{csvsimple}
\usetikzlibrary{arrows,intersections,math}
\begin{document}
\def\skala{1}
\begin{tikzpicture}[>=latex,thick,scale=\skala]

\def\punkt#1#2#3{
	\fill[color=white] #1 circle[radius=0.10];
	\fill[color=#2] #1 circle[radius=0.13];
	\node[color=white] at #1 {$\scriptstyle #3$};
}

\def\xs{2.5}
\def\ys{1}

\foreach \x in {0,...,5}{
	\draw[color=red,line width=0.5pt] 
		({\x*\xs},{-0.7*\ys}) -- ({\x*\xs},{-6.5*\ys});
}

\def\dotradius{0.04}

\def\dotrow#1#2{
	\punkt{({#1*\xs},{-1*\ys})}{#2}{1}
	\punkt{({#1*\xs},{-2*\ys})}{#2}{2}
	\punkt{({#1*\xs},{-3*\ys})}{#2}{3}
	\punkt{({#1*\xs},{-4*\ys})}{#2}{4}
	\fill[color=#2] ({#1*\xs},{-5*\ys-0.3}) circle[radius=\dotradius];
	\fill[color=#2] ({#1*\xs},{-5*\ys-0.15}) circle[radius=\dotradius];
	\fill[color=#2] ({#1*\xs},{-5*\ys}) circle[radius=\dotradius];
	\fill[color=#2] ({#1*\xs},{-5*\ys+0.15}) circle[radius=\dotradius];
	\fill[color=#2] ({#1*\xs},{-5*\ys+0.3}) circle[radius=\dotradius];
	\punkt{({#1*\xs},{-6*\ys})}{#2}{s}
}

\def\fan#1#2{
	\foreach \x in {1,2,3,4,6}{
		\foreach \y in {1,2,3,4,6}{
			\draw[->,shorten >= 2mm,shorten <= 2mm,color=#2]
				({#1*\xs},{-\x*\ys})
				--
				({(#1+1)*\xs},{-\y*\ys});
		}
	}
}

\begin{scope}
\clip (-0.5,{-6.5*\ys}) rectangle ({5*\xs+0.5},-0.5);
\fan{-1}{gray}
\fan{0}{gray}
\fan{1}{gray}
\fan{2}{black}
\fan{3}{gray}
\fan{4}{gray}
\fan{5}{gray}
\end{scope}

\dotrow{0}{gray}
\dotrow{1}{gray}
\dotrow{2}{black}
\dotrow{3}{black}
\dotrow{4}{gray}
\dotrow{5}{gray}

\def\ty{-0.5}
\node[color=gray] at ({0.5*\xs},{\ty*\ys}) {$T(n-1,n-2)$};
\node[color=gray] at ({1.5*\xs},{\ty*\ys}) {$T(n,n-1)$};
\node[color=black] at ({2.5*\xs},{\ty*\ys}) {$T(n+1,n)$};
\node[color=gray] at ({3.5*\xs},{\ty*\ys}) {$T(n+2,n+1)$};
\node[color=gray] at ({4.5*\xs},{\ty*\ys}) {$T(n+3,n+2)$};

\draw[->,color=red] (-0.7,{-6.5*\ys}) -- ({5*\xs+0.7},{-6.5*\ys}) coordinate[label={$t$}];

\foreach \x in {0,...,5}{
	\draw[color=red]
		({\x*\xs},{-6.5*\ys-0.05})
		--
		({\x*\xs},{-6.5*\ys+0.05});
}
\node[color=red] at ({0*\xs},{-6.5*\ys}) [below] {$n-2\mathstrut$};
\node[color=red] at ({1*\xs},{-6.5*\ys}) [below] {$n-1\mathstrut$};
\node[color=red] at ({2*\xs},{-6.5*\ys}) [below] {$n\mathstrut$};
\node[color=red] at ({3*\xs},{-6.5*\ys}) [below] {$n+1\mathstrut$};
\node[color=red] at ({4*\xs},{-6.5*\ys}) [below] {$n+2\mathstrut$};
\node[color=red] at ({5*\xs},{-6.5*\ys}) [below] {$n+3\mathstrut$};

\end{tikzpicture}
\end{document}

