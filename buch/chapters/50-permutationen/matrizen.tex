%
% permutationsmatrizen.tex -- Permutationsmatrizen
%
% (c) 2020 Prof Dr Andreas Müller, Hochschule Rapperswil
%
\section{Permutationsmatrizen
\label{buch:section:permutationsmatrizen}}
\rhead{Permutationsmatrizen}
Die Eigenschaft, dass eine Vertauschung das Vorzeichen kehrt, ist
eine wohlebekannte Eigenschaft der Determinanten.
In diesem Abschnitt soll daher eine Darstellung von Permutationen
als Matrizen gezeigt werden und die Verbindung zwischen dem
Vorzeichen einer Permutation und der Determinanten hergestellt
werden.

\subsection{Matrizen}
Gegeben sei jetzt eine Permutation $\sigma\in S_n$. 
Aus $\sigma$ lässt sich eine lineare Abbildung $\Bbbk^n\to\Bbbk^n$
konstruieren, die die Standardbasisvektoren permutiert, also
\[
f_{\sigma}\colon
\Bbbk^n \to \Bbbk^n
:
\left\{
\begin{aligned}
e_1&\mapsto e_{\sigma(1)} \\
e_2&\mapsto e_{\sigma(2)} \\
\vdots&\\
e_n&\mapsto e_{\sigma(n)}
\end{aligned}
\right.
\]
Die Matrix $P_\sigma$ der linearen Abbildung $f_{\sigma}$ hat in Spalte $i$
genau eine $1$ in der Zeile $\sigma(i)$, also
\[
(P_\sigma)_{ij} = \delta_{j\sigma(i)}.
\]

\begin{beispiel}
Die zur Permutation
\[
\begin{pmatrix}
1&2&3&4&5&6\\
2&1&3&5&6&4
\end{pmatrix}
\]
gehörige lineare Abbildung $f_\sigma$ hat die Matrix
\[
A_\sigma
=
\begin{pmatrix}
0&1&0&0&0&0\\
1&0&0&0&0&0\\
0&0&1&0&0&0\\
0&0&0&0&0&1\\
0&0&0&1&0&0\\
0&0&0&0&1&0
\end{pmatrix}
\qedhere
\]
\end{beispiel}

\begin{definition}
Eine Permutationsmatrix ist eine Matrix $P\in M_n(\Bbbk)$ 
derart, die in jeder Zeile und Spalte genau eine $1$ enthalten ist,
während alle anderen Matrixelemente $0$ sind.
\end{definition}

Es ist klar, dass aus einer Permutationsmatrix auch die Permutation
der Standardbasisvektoren abgelesen werden kann.
Die Verknüpfung von Permutationen wird zur Matrixmultiplikation
von Permutationsmatrizen, die Zuordnung $\sigma\mapsto P_\sigma$
ist also ein Homomorphismus
$
S_n \to M_n(\Bbbk^n),
$
es ist $P_{\sigma_1\sigma_2}=P_{\sigma_1}P_{\sigma_2}$.

\subsection{Transpositionen}
Transpositionen sind Permutationen, die genau zwei Elemente von $[n]$
vertauschen.
Wir ermitteln jetzt die Permutationsmatrix der Transposition $\tau=\tau_{ij}$
\[
P_{\tau_{ij}}
=
\begin{pmatrix}
1&      & &      &     &      & &      & \\
 &\ddots& &      &     &      & &      & \\
 &      &1&      &     &      & &      & \\
 &      & &0     &\dots&1     & &      & \\
 &      & &\vdots&     &\vdots& &      & \\
 &      & &1     &\dots&0     & &      & \\
 &      & &      &     &      &1&      & \\
 &      & &      &     &      & &\ddots& \\
 &      & &      &     &      & &      &1
\end{pmatrix}
\qedhere
\]

Die Permutation $\sigma$ mit dem Zyklus $1\to 2\to\dots\to l-1\to l\to 1$
der Länge $l$ kann aus aufeinanderfolgenden Transpositionen zusammengesetzt
werden, die zugehörigen Permutationsmatrizen sind
\begin{align*}
P_\sigma
&=
P_{\tau_{12}}
P_{\tau_{23}}
P_{\tau_{34}}\dots
P_{\tau_{l-2,l-1}}
P_{\tau_{l-1,l}}
\\
&=
\begin{pmatrix}
0&1&0&0&\dots\\
1&0&0&0&\dots\\
0&0&1&0&\dots\\
0&0&0&1&\dots\\
\vdots&\vdots&\vdots&\vdots&\ddots
\end{pmatrix}
\begin{pmatrix}
1&0&0&0&\dots\\
0&0&1&0&\dots\\
0&1&0&0&\dots\\
0&0&0&1&\dots\\
\vdots&\vdots&\vdots&\vdots&\ddots
\end{pmatrix}
\begin{pmatrix}
1&0&0&0&\dots\\
0&1&0&0&\dots\\
0&0&0&1&\dots\\
0&0&1&0&\dots\\
\vdots&\vdots&\vdots&\vdots&\ddots
\end{pmatrix}
\dots
\\
&=
\begin{pmatrix}
0&0&1&0&\dots\\
1&0&0&0&\dots\\
0&1&0&0&\dots\\
0&0&0&1&\dots\\
\vdots&\vdots&\vdots&\vdots&\ddots
\end{pmatrix}
\begin{pmatrix}
1&0&0&0&\dots\\
0&1&0&0&\dots\\
0&0&0&1&\dots\\
0&0&1&0&\dots\\
\vdots&\vdots&\vdots&\vdots&\ddots
\end{pmatrix}
\dots
\\
&=
\begin{pmatrix}
0&0&0&1&\dots\\
1&0&0&0&\dots\\
0&1&0&0&\dots\\
0&0&1&0&\dots\\
\vdots&\vdots&\vdots&\vdots&\ddots
\end{pmatrix}
\\
&\vdots\\
&=
\begin{pmatrix}
0&0&0&0&\dots&0&1\\
1&0&0&0&\dots&0&0\\
0&1&0&0&\dots&0&0\\
0&0&1&0&\dots&0&0\\
\vdots&\vdots&\vdots&\vdots&\ddots&\vdots&\vdots\\
0&0&0&0&\dots&1&0
\end{pmatrix}
\end{align*}

\subsection{Determinante und Vorzeichen}
Die Transpositionen haben Permutationsmatrizen, die aus der Einheitsmatrix
entstehen, indem genau zwei Zeilen vertauscht werden.
Die Determinante einer solchen Permutationsmatrix ist
\[
\det P_{\tau} = - \det E = -1 = \operatorname{sgn}(\tau).
\]
Nach der Produktregel für die Determinante folgt für eine Darstellung
der Permutation $\sigma=\tau_1\dots\tau_l$ als Produkt von Transpositionen,
dass
\[
\det P_{\sigma}
=
\det P_{\tau_1} \dots \det P_{\tau_l}
=
(-1)^l
=
\operatorname{sgn}(\sigma).
\]
Das Vorzeichen einer Permutation ist also identisch mit der Determinante
der zugehörigen Permutationsmatrix.


