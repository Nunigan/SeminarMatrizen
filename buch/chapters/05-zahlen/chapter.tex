%
% chapter.tex -- Kapitel mit den Grunddefinition und Notationen
%
% (c) 2020 Prof Dr Andreas Müller, OST Ostschweizer Fachhochschule
%
\chapter{Zahlen
\label{buch:chapter:zahlen}}
\lhead{Zahlen}
\rhead{}

Das Thema dieses Buches ist die Konstruktion interessanter 
mathematischer Objekte mit Hilfe von Matrizen.
Die Einträge dieser Matrizen sind natürlich Zahlen.
Wir wollen von diesen grundlegenden Bausteinen ausgehen.
Dies schliesst natürlich nicht aus, dass man auch Zahlenmengen
mit Hilfe von Matrizen beschreiben kann, wie wir es später für die
komplexen Zahlen machen werden.

In diesem Kapitel sollen daher die Eigenschaften der bekannten
Zahlensysteme der natürlichen, ganzen, rationalen, reellen und
komplexen Zahlen nochmals in einer Übersicht zusammengetragen
werden.
Dabei wird besonderes Gewicht darauf gelegt, wie in jedem Fall
einerseits neue Objekte postuliert, andererseits
aber auch konkrete Objekte konstruiert werden können.

%
% natuerlich.tex
%
% (c) 2021 Prof Dr Andreas Müller, OST Ostschweizer Fachhochschule
%
% !TeX spellcheck = de_CH
\section{Natürliche Zahlen
\label{buch:section:natuerliche-zahlen}}
\rhead{Natürliche Zahlen}
Die natürlichen Zahlen sind die Zahlen, mit denen wir zählen.
\index{natürliche Zahlen}%
\index{$\mathbb{N}$}%
Sie abstrahieren das Konzept der Anzahl der Elemente einer endlichen
Menge.
Da die leere Menge keine Elemente hat, muss die Menge der natürlichen
Zahlen auch die Zahl $0$ enthalten.
Wir schreiben
\[
\mathbb{N}
=
\{
0,1,2,3,\dots
\}.
\]

\subsubsection{Peano-Axiome}
Man kann den Zählprozess durch die folgenden Axiome von Peano beschreiben:
\index{Peano-Axiome}%
\begin{enumerate}
\item $0\in\mathbb N$.
\item Jede Zahl $n\in \mathbb{N}$ hat einen {\em Nachfolger}
$n'\in \mathbb{N}$.
\index{Nachfolger}%
\item $0$ ist nicht Nachfolger einer Zahl.
\item Wenn zwei Zahlen $n,m\in\mathbb{N}$ den gleichen Nachfolger haben,
$n'=m'$, dann sind sie gleich $n=m$.
\item Enthält eine Menge $X$ die Zahl $0$ und mit jeder Zahl auch ihren
Nachfolger, dann ist $\mathbb{N}\subset X$.
\end{enumerate}

\subsubsection{Vollständige Induktion}
Es letzte Axiom formuliert das Prinzip der vollständigen Induktion.
Um eine Aussage $P(n)$ für alle natürlichen Zahlen $n$
mit vollständiger Induktion zu beweisen, bezeichnet man mit
$X$ die Menge aller Zahlen, für die $P(n)$ wahr ist.
Die Induktionsverankerung beweist, dass $P(0)$ wahr ist, dass also $0\in X$.
Der Induktionsschritt beweist, dass mit einer Zahl $n\in X$ auch der
Nachfolger $n'\in X$ ist.
Nach dem letzten Axiom ist $\mathbb{N}\subset X$, oder anders ausgedrückt,
die Aussage $P(n)$ ist wahr für jede natürliche Zahl.

\subsubsection{Addition}
Aus der Nachfolgereigenschaft lässt sich durch wiederholte Anwendung
die vertrautere Addition konstruieren.
\index{Addition!in $\mathbb{N}$}%
Um die Zahl $n\in\mathbb{N}$ um $m\in\mathbb{N}$ zu vermehren, also
$n+m$ auszurechnen, kann man rekursive Regeln
\begin{align*}
n+0&=n\\
n+m'&=(n+m)'
\end{align*}
festlegen.
Nach diesen Regeln ist
\[
5+3
=
5+2'
=
(5+2)'
=
(5+1')'
=
((5+1)')'
=
((5+0')')'
=
(((5)')')'.
\]
Dies ist genau die Art und Weise, wie kleine Kinder Rechnen lernen.
Sie Zählen von $5$ ausgehend um $3$ weiter.
Der dritte Nachfolger von $5$ heisst üblicherweise $8$.

Die algebraische Struktur, die hier konstruiert worden ist, heisst
eine Halbgruppe.
Allerdings kann man darin zum Beispiel nur selten Gleichungen
lösen, zum Beispiel hat $3+x=1$ keine Lösung.
Die Addition ist nicht immer umkehrbar.

\subsubsection{Multiplikation}
Es ist klar, dass auch die Multiplikation definiert werden kann, 
sobald die Addition definiert ist.
Die Rekursionsformeln
\begin{align}
n\cdot 0 &= 0 \notag \\
n\cdot m' &= n\cdot m + n
\label{buch:zahlen:multiplikation-rekursion}
\end{align}
legen jedes Produkt von natürlichen Zahlen fest, zum Beispiel
\[
5\cdot 3
=
5\cdot 2'
=
5\cdot 2 + 5
=
5\cdot 1' + 5
=
5\cdot 1 + 5 + 5
=
5\cdot 0' + 5 + 5
=
5\cdot 0 + 5 + 5 + 5
=
5 + 5 + 5.
\]
Doch auch bezüglich der Multiplikation ist $\mathbb{N}$ unvollständig,
die Beispielgleichung $3x=1$ hat keine Lösung in $\mathbb{N}$.

\subsubsection{Rechenregeln}
Aus den Definitionen lassen sich auch die Rechenregeln ableiten,
die man für die alltägliche Rechnung braucht.
Zum Beispiel kommt es nicht auf die Reihenfolge der Summanden
oder Faktoren an. 
Das {\em Kommutativgesetz} besagt
\[
a+b=b+a
\qquad\text{und}\qquad
a\cdot b = b\cdot a.
\]
\index{Kommutativgesetz}%
Die Kommutativität der Addition werden wir auch in allen weiteren
Konstruktionen voraussetzen.
Die Kommutativität des Produktes ist allerdings weniger selbstverständlich
und wird beim Matrizenprodukt nur noch für spezielle Faktoren zutreffen.

Eine Summe oder ein Produkt mit mehr als zwei Summanden bzw.~Faktoren
kann in jeder beliebigen Reihenfolge ausgewertet werden,
\[
(a+b)+c
=
a+(b+c)
\qquad\text{und}\qquad
(a\cdot b)\cdot c
=
a\cdot (b\cdot c)
\]
dies ist das Assoziativgesetz.
Es gestattet auch eine solche Summe oder ein solches Produkt einfach
als $a+b+c$ bzw.~$a\cdot b\cdot c$ zu schreiben, da es ja keine Rolle
spielt, in welcher Reihenfolge man die Teilprodukte berechnet.

Die Konstruktion der Multiplikation als iterierte Addition mit Hilfe
der Rekursionsformel \eqref{buch:zahlen:multiplikation-rekursion}
hat auch zur Folge, dass die {\em Distributivgesetze}
\[
a\cdot(b+c) = ab+ac
\qquad\text{und}\qquad
(a+b)c = ac+bc
\]
gelten.
Bei einem nicht-kommutativen Produkt ist es hierbei notwendig,
zwischen Links- und Rechts-Distributivgesetz zu unterscheiden.

Die Distributivgesetze drücken die wohlbekannte Regel des
Ausmultiplizierens aus.
Ein Distributivgesetz ist also grundlegend dafür, dass man mit den
Objekten so rechnen kann, wie man das in der elementaren Algebra 
gelernt hat.
Auch die Distributivgesetze sind daher Rechenregeln, die wir in
Zukunft immer dann fordern werden, wenn Addition und Multiplikation
definiert sind.
Sie gelten immer für Matrizen.

\subsubsection{Teilbarkeit}
Die Lösbarkeit von Gleichungen der Form $ax=b$ mit $a,b\in\mathbb{N}$
gibt Anlass zum sehr nützlichen Konzept der Teilbarkeit.
\index{Teilbarkeit}%
Die Zahl $b$ heisst teilbar durch $a$, wenn die Gleichung $ax=b$ eine
Lösung in $\mathbb{N}$ hat.
\index{teilbar}%
Jede natürlich Zahl $n$ ist durch $1$ und durch sich selbst teilbar,
denn $n\cdot 1 = n$.
Andere Teiler sind dagegen nicht selbstverständlich.
Die Zahlen
\[
\mathbb{P}
=
\{2,3,5,7,11,13,17,19,23,29,\dots\}
\]
haben keine weiteren Teiler. Sie heissen {\em Primzahlen}.
\index{Primzahl}%
Die Menge der natürlichen Zahlen ist die naheliegende Arena
für die Zahlentheorie.
\index{Zahlentheorie}%

\subsubsection{Konstruktion der natürlichen Zahlen aus der Mengenlehre}
Die Peano-Axiome postulieren, dass es natürliche Zahlen gibt.
Es werden keine Anstrengungen unternommen, die natürlichen Zahlen
aus noch grundlegenderen mathematischen Objekten zu konstruieren.
Die Mengenlehre bietet eine solche Möglichkeit.

Da die natürlichen Zahlen das Konzept der Anzahl der Elemente einer
Menge abstrahieren, gehört die leere Menge zur Zahl $0$.
Die Zahl $0$ kann also durch die leere Menge $\emptyset = \{\}$
wiedergegeben werden.

Der Nachfolger muss jetzt als eine Menge mit einem Element konstruiert
werden.
Das einzige mit Sicherheit existierende Objekt, das für diese Menge
zur Verfügung steht, ist $\emptyset$.
Zur Zahl $1$ gehört daher die Menge $\{\emptyset\}$, eine Menge mit
genau einem Element.
Stellt die Menge $N$ die Zahl $n$ dar, dann können wir die zu $n+1$
gehörige Menge $N'$ dadurch konstruieren, dass wir zu den Elemente
von $N$ ein zusätzliches Element hinzufügen, das noch nicht in $N$ ist,
zum Beispiel $\{N\}$:
\[
N' = N \cup \{ N \}.
\]

Die natürlichen Zahlen existieren also, wenn wir akzeptieren, dass es
Mengen gibt.
Die natürlichen Zahlen sind dann nacheinander die Mengen
\begin{align*}
0 &= \emptyset 
\\
1 &= 0 \cup \{0\} = \emptyset \cup \{0\} = \{0\}
\\
2 &= 1 \cup \{1\} = \{0\}\cup\{1\} = \{0,1\}
\\
3 &= 2 \cup \{2\} = \{0,1\}\cup \{2\} = \{0,1,2\}
\\
&\phantom{n}\vdots
\\
n+1&= n \cup \{n\} = \{0,\dots,n-1\} \cup \{n\} = \{0,1,\dots,n\}
\\
&\phantom{n}\vdots
\end{align*}

\subsubsection{Natürliche Zahlen als Äquivalenzklassen}
Im vorangegangenen Abschnitt haben wir die natürlichen Zahlen aus
der leeren Menge schrittweise sozusagen ``von unten'' aufgebaut.
Wir können aber auch eine Sicht ``von oben'' einnehmen.
Dazu definieren wir, was eine endliche Menge ist und was es heisst,
dass endliche Mengen gleiche Mächtigkeit haben.

\begin{definition}
Eine Menge $A$ heisst {\em endlich}, wenn es jede injektive Abbildung
$A\to A$ auch surjektiv ist.
\index{endlich}%
Zwei endliche Mengen $A$ und $B$ heissen {\em gleich mächtig},
\index{gleich mächtig}%
in Zeichen $A\sim B$, wenn es eine Bijektion
$A\to B$ gibt.
\end{definition}

Der Vorteil dieser Definition ist, dass sie die früher definierten 
natürlichen Zahlen nicht braucht, diese werden jetzt erst konstruiert.
Dazu fassen wir in der Menge aller endlichen Mengen die gleich mächtigen
Mengen zusammen, bilden also die Äquivalenzklassen der Relation $\sim$.

Der Vorteil dieser Sichtweise ist, dass die natürlichen Zahlen ganz
explizit als die Anzahlen von Elementen einer endlichen Menge entstehen.
Eine natürlich Zahl ist also eine Äquivalenzklasse
$\llbracket A\rrbracket$, die alle endlichen Mengen enthält,  die die
gleiche Mächtigkeit wie $A$ haben.
Zum Beispiel gehört dazu auch die Menge, die im vorangegangenen
Abschnitt aus der leeren Menge aufgebaut wurde.

Die Mächtigkeit einer endlichen Menge $A$ ist die Äquivalenzklasse, in der
die Menge drin ist: $|A| = \llbracket A\rrbracket\in \mathbb{N}$ nach 
Konstruktion von $\mathbb{N}$.
Aus logischer Sicht etwas problematisch ist allerdings, dass wir 
von der ``Menge aller endlichen Mengen'' sprechen ohne uns zu versichern,
dass dies tatsächlich eine zulässige Konstruktion ist.


%
% ganz.tex
%
% (c) 2021 Prof Dr Andreas Müller, Hochschule Rapperswil
%
\begin{frame}[t]
\frametitle{Ganze Zahlen: Gruppe}
\vspace{-20pt}
\begin{columns}[t,onlytextwidth]
\begin{column}{0.48\textwidth}
\setlength{\abovedisplayskip}{5pt}
\setlength{\belowdisplayskip}{5pt}
\begin{block}{Subtrahieren}
Nicht für alle $a,b\in \mathbb{N}$ hat die 
Gleichung
\[
a+x=b
\uncover<2->{
\quad
\Rightarrow
\quad
x=b-a}
\]
eine Lösung in $\mathbb{N}$\uncover<2->{, nämlich wenn $a>b$}%
\end{block}
\uncover<3->{%
\begin{block}{Ganze Zahlen = Paare}
Idee: $b-a = (b,a)$
\begin{enumerate}
\item<4-> $(b,a)=\mathbb{N}\times\mathbb{N}$
\item<5-> Äquivalenzrelation
\[
(b,a)\sim (d,c)
\ifthenelse{\boolean{presentation}}{
\only<6>{\Leftrightarrow
\text{``\strut}
b-a=c-d
\text{\strut''}}}{}
\only<7->{
\Leftrightarrow
b+d=c+a}
\]
\end{enumerate}
\vspace{-10pt}
\uncover<8->{%
Ganze Zahlen:
\(
\mathbb{Z}
=
\mathbb{N}\times\mathbb{N}/\sim
\)}
\\
\uncover<9->{%
$z\in\mathbb{Z}$, $z=\mathstrut$ Paare $(u,v)$ mit 
``gleicher Differenz''}
\uncover<10->{%
$\Rightarrow$ alle Differenzen in $\mathbb{Z}$}
\end{block}}
\end{column}
\begin{column}{0.48\textwidth}
\setlength{\abovedisplayskip}{5pt}
\setlength{\belowdisplayskip}{5pt}
\uncover<11->{%
\begin{block}{Gruppe}
Monoid $\ifthenelse{\boolean{presentation}}{\only<11>{\mathbb{Z}}}{}\only<12->{G}$ mit inversem Element
\[
a\in \ifthenelse{\boolean{presentation}}{\only<11>{\mathbb{Z}}}{}\only<12->{G}
\Rightarrow
\ifthenelse{\boolean{presentation}}{\only<11>{-a\in\mathbb{Z}}}{}\only<12->{a^{-1}\in G}
\text{ mit }
\ifthenelse{\boolean{presentation}}{
\only<11>{
a+(-a)=0
}}{}
\only<12->{
\left\{
\begin{aligned}
aa^{-1}&=e
\\
a^{-1}a&=e
\end{aligned}
\right.
}
\]
\end{block}}
\vspace{-15pt}
\uncover<13->{%
\begin{block}{Abelsche Gruppe}
Verknüpfung ist kommutativ:
\[
a+b=b+a
\]
\end{block}}
\vspace{-12pt}
\uncover<14->{%
\begin{block}{Beispiele}
\begin{itemize}
\item<15-> Brüche, reelle Zahlen
\item<16-> invertierbare Matrizen: $\operatorname{GL}_n(\mathbb{R})$
\item<17-> Drehmatrizen: $\operatorname{SO}(n)$
\item<18-> Matrizen mit Determinante $1$: $\operatorname{SL}_n(\mathbb R)$
\end{itemize}
\end{block}}
\end{column}
\end{columns}
\end{frame}

%
% rational.tex -- rationale Zahlen
%
% (c) 2021 Prof Dr Andreas Müller, OST Ostschweizer Fachhochschule
%
% !TeX spellcheck = de_CH
\section{Rationale Zahlen
\label{buch:section:rationale-zahlen}}
\rhead{Rationale Zahlen}
In den ganzen Zahlen sind immer noch nicht alle linearen Gleichungen
lösbar, es gibt keine ganze Zahl $x$ mit $3x=1$.
Die nötige Erweiterung der ganzen Zahlen lernen Kinder noch bevor sie
die negativen Zahlen kennenlernen.

Wir können hierbei denselben Trick anwenden,
wie schon beim Übergang von den natürlichen zu den ganzen Zahlen.
Wir kreieren wieder Paare $(z, n)$, deren Elemente nennen wir \emph{Zähler} und
\emph{Nenner}, wobei $z, n \in \mathbb Z$ und zudem $n \ne 0$.
Die Rechenregeln für Addition und Multiplikation lauten
\[
(a, b) + (c, d)
=
(ad + bc, bd)
\qquad \text{und} \qquad
(a, b) \cdot (c, d)
=
(ac, bd)
.
\]
Die ganzen Zahlen lassen sich als in dieser Darstellung als
$z \mapsto (z, 1)$ einbetten.

Ähnlich wie schon bei den ganzen Zahlen ist diese Darstellung
aber nicht eindeutig.
Zwei Paare sind äquivalent, wenn sich deren beide Elemente um denselben Faktor
unterscheiden,
\[
(a, b)
\sim
(c, d)
\quad \Leftrightarrow \quad
\exists \lambda \in \mathbb Z \colon
\lambda a = c
\wedge
\lambda b = d
.
\]
Dass es sich hierbei wieder um eine Äquivalenzrelation handelt, lässt sich
einfach nachprüfen.

Durch die neuen Regen gibt es nun zu jedem Paar $(a, b)$ mit $a \ne 0$
ein Inverses $(b, a)$ bezüglich der Multiplikation,
wie man anhand der folgenden Rechnung sieht,
\[
(a, b) \cdot (b, a)
=
(a \cdot b, b \cdot a)
=
(a \cdot b, a \cdot b)
\sim
(1, 1)
.
\]

\subsubsection{Brüche}
Rationale Zahlen sind genau die Äquivalenzklassen dieser Paare $(a, b)$ von
ganzen Zahlen $a$ und $b\ne 0$.
Da diese Schreibweise recht unhandlich ist, wird normalerweise die Notation
als Bruch $\frac{a}{b}$ verwendet.
Die Rechenregeln werden dadurch zu den wohlvertrauten
\[
\frac{a}{b}+\frac{c}{d}
=
\frac{ad+bc}{bd},
\qquad\text{und}\qquad
\frac{a}{b}\cdot\frac{c}{d}
=
\frac{ac}{bd}
\]
und die speziellen Brüche $\frac{0}{b}$ und $\frac{1}{1}$ erfüllen die
Regeln
\[
\frac{a}{b}+\frac{0}{d} = \frac{ad}{bd} \sim \frac{a}{b},
\qquad
\frac{a}{b}\cdot \frac{0}{c} = \frac{0}{bc}
\qquad\text{und}\qquad
\frac{a}{b}\cdot \frac{1}{1} = \frac{a}{b}.
\]
Wir sind uns gewohnt, die Brüche $\frac{0}{b}$ mit der Zahl $0$ und
$\frac{1}{1}$ mit der Zahl $1$ zu identifizieren.

\subsubsection{Kürzen}
Wie bei den ganzen Zahlen entstehen durch die Rechenregeln viele Brüche,
denen wir den gleichen Wert zuordnen möchten.
Zum Beispiel folgt
\[
\frac{ac}{bc} - \frac{a}{b} 
=
\frac{abc-abc}{b^2c}
=
\frac{0}{b^2c},
\]
wir müssen also die beiden Brüche als gleichwertig betrachten.
Allgemein gelten die zwei Brüche $\frac{a}{b}$ und $\frac{c}{d}$
als äquivalent, wenn $ad-bc= 0$ gilt.
Dies ist gleichbedeutend mit der früher definierten Äquivalenzrelation
und bestätigt, dass die beiden Brüche
\[
\frac{ac}{bc} 
\qquad\text{und}\qquad
\frac{a}{b}
\]
als gleichwertig zu betrachten sind.
Der Übergang von links nach rechts heisst {\em Kürzen},
\index{Kürzen}%
der Übergang von rechts nach links heisst {\em Erweitern}.
\index{Erweitern}%
Eine rationale Zahl ist also eine Menge von Brüchen, die durch
Kürzen und Erweitern ineinander übergeführt werden können.

Die Menge der Äquivalenzklassen von Brüchen ist die Menge $\mathbb{Q}$
der rationalen Zahlen.
In $\mathbb{Q}$ sind Addition, Subtraktion und Multiplikation mit den
gewohnten Rechenregeln, die bereits in $\mathbb{Z}$ gegolten haben,
uneingeschränkt möglich.

\subsubsection{Kehrwert}
Zu jedem Bruch $\frac{a}{b}$ lässt sich der Bruch $\frac{b}{a}$,
der sogenannte {\em Kehrwert}
\index{Kehrwert}
konstruieren.
Er hat die Eigenschaft, dass
\[
\frac{a}{b}\cdot\frac{b}{a}
=
\frac{ab}{ba}
=
1
\]
gilt.
Der Kehrwert ist also das multiplikative Inverse, jede von $0$ verschiedene
rationale Zahl hat eine Inverse.

\subsubsection{Lösung von linearen Gleichungen}
Mit dem Kehrwert lässt sich jetzt jede lineare Gleichung lösen.
\index{lineares Gleichungssystem}%
Die Gleichung $ax=b$ hat die Lösung
\[
ax = \frac{a}{1} \frac{u}{v} = \frac{b}{1}
\qquad\Rightarrow\qquad
\frac{1}{a}
 \frac{a}{1} \frac{u}{v} = \frac{1}{a}\frac{b}{1} 
\qquad\Rightarrow\qquad
\frac{u}{v} = \frac{b}{a}.
\]
Dasselbe gilt auch für rationale Koeffizienten $a$ und $b$.
In der Menge $\mathbb{Q}$ kann man also beliebige lineare Gleichungen
lösen.

\subsubsection{Körper}
$\mathbb{Q}$ ist ein Beispiel für einen sogenannten {\em Körper}, 
\index{Körper}%
in dem die arithmetischen Operationen Addition, Subtraktion, Multiplikation
und Division möglich sind mit der einzigen Einschränkung, dass nicht durch
$0$ dividiert werden kann.
Körper sind die natürliche Bühne für die lineare Algebra, da sich lineare
Gleichungssysteme ausschliesslich mit den Grundoperation lösen lassen.

Wir werden im Folgenden für verschiedene Anwendungszwecke weitere Körper
konstruieren, zum Beispiel die reellen Zahlen $\mathbb{R}$ und die
rationalen Zahlen $\mathbb{C}$.
Wann immer die Wahl des Körpers keine Rolle spielt, werden wir den
Körper mit $\Bbbk$ bezeichnen.
\index{$\Bbbk$}%




%
% reell.tex -- reelle Zahlen
%
% (c) 2021 Prof Dr Andreas Müller, OST Ostschweizer Fachhochschule
%
\section{Reelle Zahlen
\label{buch:section:reelle-zahlen}}
\rhead{Reelle Zahlen}
In den rationalen Zahlen lassen sich algebraische Gleichungen höheren
Grades immer noch nicht lösen.
Dass die Gleichung $x^2=2$ keine rationale Lösung hat, ist schon den
Pythagoräern aufgefallen.
Die geometrische Intuition der Zahlengeraden führt uns dazu, nach
Zahlen zu suchen, die gute Approximationen für $\sqrt{2}$ sind.
Wir können zwar keinen Bruch angeben, dessen Quadrat $2$ ist, aber
wenn es eine Zahl $\sqrt{2}$ mit dieser Eigenschaft gibt, dann können
wir dank der Ordnungsrelation feststellen, dass sie in all den folgenden,
kleiner werdenden Intervallen
\[
\biggl[1,\frac32\biggr],\;
\biggl[\frac75,\frac{17}{12}\biggr],\;
\biggl[\frac{41}{29},\frac{99}{70}\biggr],\;
\biggl[\frac{239}{169},\frac{577}{408}\biggr],\;
\dots
\]
enthalten sein muss\footnote{Die Näherungsbrüche konvergieren sehr
schnell, sie sind mit der sogenannten Kettenbruchentwicklung der
Zahl $\sqrt{2}$ gewonnen worden.}.
Jedes der Intervalle enthält auch das nachfolgende Intervall, und
die intervalllänge konvergiert gegen 0.
Eine solche \emph{Intervallschachtelung} beschreibt also genau eine Zahl,
aber möglicherweise keine, die sich als Bruch schreiben lässt.

Die Menge $\mathbb{R}$ der reellen Zahlen kann man auch als Menge
aller Cauchy-Folgen $(a_n)_{n\in\mathbb{N}}$ betrachten.
Eine Folge ist eine Cauchy-Folge, wenn es für jedes $\varepsilon>0$
eine Zahl $N(\varepsilon)$ gibt derart, dass $|a_n-a_m|<\varepsilon$
für $n,m>N(\varepsilon)$.
Ab einer geeigneten Stelle $N(\varepsilon)$ sind die Folgenglieder also
mit Genauigkeit $\varepsilon$ nicht mehr unterscheidbar.

Nicht jede Cauchy-Folge hat eine rationale Zahl als Grenzwert.
Da wir für solche Folgen noch keine Zahlen als Grenzwerte haben,
nehmen wir die Folge als eine mögliche Darstellung der Zahl.
Die Folge kann man ja auch verstehen als eine Vorschrift, wie man
Approximationen der Zahl berechnen kann.

Zwei verschiedene Cauchy-Folgen $(a_n)_{n\in\mathbb{N}}$ und
$(b_n)_{n\in\mathbb{N}}$ 
können den gleichen Grenzwert haben.
So sind 
\[
\begin{aligned}
a_n&\colon&&
1,\frac32,\frac75,\frac{17}{12},\frac{41}{29},\frac{99}{70},\frac{239}{169},
\frac{577}{408},\dots
\\
b_n&\colon&&
1,1.4,1.41,1.412,1.4142,1.41421,1.414213,1.4142135,\dots
\end{aligned}
\]
beide Folgen, die die Zahl $\sqrt{2}$ approximieren.
Im Allgemeinen tritt dieser Fall ein, wenn $|a_n-b_n|$ eine
Folge mit Grenzwert $0$ oder Nullfolge ist.
Eine reelle Zahl ist also die Menge aller rationalen Cauchy-Folgen,
deren Differenzen Nullfolgen sind.

Die Menge $\mathbb{R}$ der reellen Zahlen kann man also ansehen
als bestehend aus Mengen von Folgen, die alle den gleichen Grenzwert
haben.
Die Rechenregeln der Analysis 
\[
\lim_{n\to\infty} (a_n + b_n)
=
\lim_{n\to\infty} a_n +
\lim_{n\to\infty} b_n
\qquad\text{und}\qquad
\lim_{n\to\infty} a_n \cdot b_n
=
\lim_{n\to\infty} a_n \cdot
\lim_{n\to\infty} b_n 
\]
stellen sicher, dass sich die Rechenoperationen von den rationalen
Zahlen auf die reellen Zahlen übertragen lassen.





%
% komplex.tex -- simpliziale Komplexe und Kettenkomplexe
%
% (c) 2021 Prof Dr Andreas Müller, OST Ostschweizer Fachhochschule
%
\section{Kettenkomplexe
\label{buch:section:komplex}}
\rhead{Kettenkomplexe}

\subsection{Randoperator von Simplexen
\label{buch:subsection:randoperator-von-simplexen}}

\subsection{Kettenkomplexe und Morphismen
\label{buch:subsection:kettenkomplex}}


%\section*{Übungsaufgaben}
%\aufgabetoplevel{chapters/05-zahlen/uebungsaufgaben}
%\begin{uebungsaufgaben}
%\end{uebungsaufgaben}

