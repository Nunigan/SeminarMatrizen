Rechnen Sie nach, dass die Matrix
\[
A
=
\begin{pmatrix}
2&1&0\\
0&2&1\\
1&0&2
\end{pmatrix}
\]
normal ist.
\begin{teilaufgaben}
\item
Berechnen Sie die Eigenwerte, indem Sie das charakteristische Polynom 
von $A$ und seine Nullstellen bestimmen.
\item
Das Polynom
\[
p(z,\overline{z})
=
\frac{(3-\sqrt{3})z\overline{z}-9(1-\sqrt{3})}{6}
\]
hat die Eigenschaft, dass 
\begin{align*}
p(\lambda,\lambda) &= |\lambda|
\end{align*}
für alle drei Eigenwerte von $A$.
Verwenden Sie dieses Polynom, um $B=|A|$ zu berechen.
\item
Überprüfen Sie Ihr Resultat, indem Sie mit einem Computeralgebra-Programm
die Eigenwerte von $B$ bestimmen.
\end{teilaufgaben}

\begin{loesung}
Die Matrix $A$ ist von der Form $2I+O$ mit $O\in\operatorname{SO}(3)$,
für solche Matrizen wurde gezeigt, dass sie normal sind.
Man kann aber auch direkt nachrechnen:
\begin{align*}
AA^t
&=
\begin{pmatrix}
2&1&0\\
0&2&1\\
1&0&2
\end{pmatrix}
\begin{pmatrix}
2&0&1\\
1&2&0\\
0&1&2
\end{pmatrix}
=
\begin{pmatrix}
5&2&2\\
2&5&2\\
2&2&5
\end{pmatrix}
\\
A^tA
&=
\begin{pmatrix}
2&0&1\\
1&2&0\\
0&1&2
\end{pmatrix}
\begin{pmatrix}
2&1&0\\
0&2&1\\
1&0&2
\end{pmatrix}
=
\begin{pmatrix}
5&2&2\\
2&5&2\\
2&2&5
\end{pmatrix}
\end{align*}
Es gilt also $AA^t=A^tA$, die Matrix ist also normal.
\begin{teilaufgaben}
\item Das charakteristische Polynom ist
\begin{align}
\chi_A(\lambda)
&=\left|
\begin{matrix}
2-\lambda &    1      &      0     \\
    0     & 2-\lambda &      1     \\
    1     &    0      & 2-\lambda
\end{matrix}
\right|
=
(2-\lambda)^3+1
\label{4005:charpoly}
\\
&=-\lambda^3 -6\lambda^2 + 12\lambda +9.
\notag
\end{align}
Mit einem Taschenrechner kann man die Nullstellen finden,
aber man kann das auch die Form \eqref{4005:charpoly}
des charakteristischen Polynoms direkt faktorisieren:
\begin{align*}
\chi_A(\lambda)
&=
(2-\lambda)^3+1
\\
&=
((2-\lambda)+1)
((2-\lambda)^2 -(2-\lambda)+1)
\\
&=
(3-\lambda)
(\lambda^2-3\lambda +4-2+\lambda +1)
\\
&=
(3-\lambda)
(\lambda^2-2\lambda +3)
\end{align*}
Daraus kann man bereits einen Eigenwert $\lambda=3$ ablesen,
die weiteren Eigenwerte sind die Nullstellen des zweiten Faktors, die
man mit der Lösungsformel für quadratische Gleichungen finden kann:
\begin{align*}
\lambda_{\pm}
&=
\frac{3\pm\sqrt{9-12}}{2}
=
\frac{3}{2} \pm\frac{\sqrt{-3}}{2}
=
\frac{3}{2} \pm i\frac{\sqrt{3}}{2}
\end{align*}
\item
Wir müssen $z=A$ und $\overline{z}=A^t$ im Polynom $p(z,\overline{z})$
substituieren und erhalten
\begin{align*}
B
&=
\frac{3-\sqrt{3}}6 \begin{pmatrix}5&2&2\\2&5&2\\2&2&5\end{pmatrix}
+\frac{\sqrt{3}-1}{2}I
\\
&=
\begin{pmatrix}
   2.1547005&  0.42264973&  0.42264973 \\
   0.4226497&  2.15470053&  0.42264973 \\
   0.4226497&  0.42264973&  2.15470053
\end{pmatrix}
\end{align*}
\item
Tatsächlich gibt die Berechnung der Eigenwerte
den einfachen Eigenwert $\mu_0=3=|\lambda_0|$
und
den doppelten Eigenwert $\mu_{\pm} = \sqrt{3}=1.7320508=|\lambda_{\pm}|$.
\qedhere
\end{teilaufgaben}
\end{loesung}
