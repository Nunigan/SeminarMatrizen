Verwenden Sie die Matrixdarstellung komplexer Zahlen, um $i^i$ zu
berechnen.

\begin{hinweis}
Verwenden Sie die Eulersche Formel um $\log J$ zu bestimmen.
\end{hinweis}

\begin{loesung}
Wir berechnen $J^J$ mit Hilfe des Logarithmus als
$J^J = \exp(J\log J)$.
Zunächst erinnern wir an die Eulersche Formel
\[
\exp tJ
=
\sum_{k=0}^\infty \frac{t^k J^k}{k!}
=
\sum_{i=0}^\infty \frac{t^{2i}(-1)^i}{(2i)!}\cdot I
+
\sum_{i=0}^\infty \frac{t^{2i+1}(-1)^i}{(2i+1)!}\cdot J
=
\cos t\cdot I
+
\sin t\cdot J.
\]
Daraus liest man ab, dass 
\[
\log \begin{pmatrix}
\cos t&-\sin t\\
\sin t& \cos t
\end{pmatrix}
=
tJ
\]
gilt.
Für die Matrix $J$ heisst das
\begin{equation}
J = \begin{pmatrix}
0&-1\\1&0
\end{pmatrix}
=
\begin{pmatrix}
\cos\frac{\pi}2&-\sin\frac{\pi}2\\
\sin\frac{\pi}2& \cos\frac{\pi}2
\end{pmatrix}
\qquad\Rightarrow\qquad
\log J = \frac{\pi}2 J.
\label{4001:logvalue}
\end{equation}
Als nächstes müssen wir $J\log J$ berechnen.
Aus \eqref{4001:logvalue} folgt
\[
J\log J = J\cdot \frac{\pi}2J = - \frac{\pi}2 \cdot I.
\]
Darauf ist die Exponentialreihe auszuwerten, also
\[
J^J
=
\exp (J\log J)
=
\exp(-\frac{\pi}2 I)
=
\exp
\begin{pmatrix}
-\frac{\pi}2&0\\
0&-\frac{\pi}2
\end{pmatrix}
=
\begin{pmatrix}
e^{-\frac{\pi}2}&0\\
0&e^{-\frac{\pi}2}
\end{pmatrix}
=
e^{-\frac{\pi}2} I.
\]
Als komplexe Zahlen ausgedrückt folgt also $i^i = e^{-\frac{\pi}2}$.
\end{loesung}
