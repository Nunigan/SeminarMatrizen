Seien $z$ und $w$  komplexe Zahlen derart, dass $z=e^w$, d.~h.~$w$ ist
ein Wert des Logarithmus von $z$.
Zeigen Sie, dass die Zahlen $w+2\pi ik$ für $k\in\mathbb Z$ ebenfalls 
Logarithmen von $z$ sind.
Dies zeigt, dass eine komlexe Zahl unendlich viele verschiedene
Logarithmen haben kann, die Logarithmusfunktion ist im Komplexen
nicht eindeutig.

\begin{loesung}
Aus der Eulerschen Formel folgt
\begin{align*}
e^{w+2\pi ik}
&=
e^w\cdot e^{2\pi ik}
=
e^w (\underbrace{\cos 2\pi k}_{\displaystyle=1} + i \underbrace{\sin 2\pi k}_{\displaystyle = 0})
=
e^w
=
z.
\qedhere
\end{align*}
\end{loesung}
