%
% wurzelapprox.tex -- template for standalon tikz images
%
% (c) 2021 Prof Dr Andreas Müller, OST Ostschweizer Fachhochschule
%
\documentclass[tikz]{standalone}
\usepackage{amsmath}
\usepackage{times}
\usepackage{txfonts}
\usepackage{pgfplots}
\usepackage{csvsimple}
\usetikzlibrary{arrows,intersections,math}
\begin{document}
\def\skala{5.7}
\begin{tikzpicture}[>=latex,thick,scale=\skala]

\definecolor{darkgreen}{rgb}{0,0.6,0}

\input{wa.tex}

\begin{scope}[xshift=-0.63cm]

\draw[->] (-0.01,0) -- (1.05,0) coordinate[label={$t$}];

\begin{scope}
	\clip (0,0) rectangle (1,1.01);
	\draw[color=blue,line width=1.6pt] \pfada;
	\draw[color=blue,line width=1.2pt] \pfadb;
	\draw[color=blue,line width=1.2pt] \pfadc;
	\draw[color=blue,line width=1.2pt] \pfadd;
	\draw[color=blue,line width=1.2pt] \pfade;
	\draw[color=blue,line width=1.2pt] \pfadf;
	\draw[color=blue,line width=1.2pt] \pfadg;
	\draw[color=blue,line width=1.2pt] \pfadh;
	\draw[color=blue,line width=1.2pt] \pfadi;
	\draw[color=blue,line width=1.2pt] \pfadj;

	\draw[color=red,line width=1.6pt]
		plot[domain=0:1.01,samples=100] ({\x*\x},{\x});
\end{scope}

\node[color=red] at (0.5,0.707) [above,rotate={atan(0.5)}] {$\sqrt{t}$};

\draw[->] (0,-0.01) -- (0,1.05) coordinate[label={right:$u_n(t)$}];

\foreach \x in {2,4,...,8}{
	\draw ({0.1*\x},-0.01) -- ({0.1*\x},0.01);
	\node at ({0.1*\x},-0.01) [below] {0.\x\strut};
	\draw (-0.01,{0.1*\x}) -- (0.01,{0.1*\x});
	\node at (-0.01,{0.1*\x}) [left] {0.\x\strut};
}
\draw (1,-0.01) -- (1,0.01);
\node at (1,-0.01) [below] {1.0\strut};
\node at (0,-0.01) [below] {0\strut};

\draw (-0.01,1) -- (0.01,1);
\node at (-0.01,1) [left] {1.0\strut};

\node[color=blue] at (1.01,0) [above left] {$u_0(t)$};
\node[color=blue] at (1,0.51) [below left,rotate={atan(0.5)}] {$u_1(t)$};
\node[color=blue] at (1,{0.86+0.03}) [below left,rotate={atan(0.86)}] {$u_2(t)$};
\node[color=blue] at (1,1.00) [below left,rotate={atan(0.5)}] {$u_3(t)$};

\end{scope}

\begin{scope}[xshift=0.63cm]

\begin{scope}
	\clip (0,0) rectangle (1,1.01);
	\draw[color=darkgreen,line width=1.2pt] \fehlera;
	\draw[color=darkgreen,line width=1.2pt] \fehlerb;
	\draw[color=darkgreen,line width=1.2pt] \fehlerc;
	\draw[color=darkgreen,line width=1.2pt] \fehlerd;
	\draw[color=darkgreen,line width=1.2pt] \fehlere;
	\draw[color=darkgreen,line width=1.2pt] \fehlerf;
	\draw[color=darkgreen,line width=1.2pt] \fehlerg;
	\draw[color=darkgreen,line width=1.2pt] \fehlerh;
	\draw[color=darkgreen,line width=1.2pt] \fehleri;
	\draw[color=darkgreen,line width=1.2pt] \fehlerj;
\end{scope}

\draw[->] (0,-0.01) -- (0,1.05) coordinate[label={right:${\color{red}\sqrt{t}}-{\color{blue}u_n(t)}$}];
\draw[->] (-0.01,0) -- (1.05,0) coordinate[label={$t$}];

\foreach \x in {2,4,...,9}{
	\draw ({0.1*\x},-0.01) -- ({0.1*\x},0.01);
	\node at ({0.1*\x},-0.01) [below] {0.\x\strut};
	\draw (-0.01,{0.1*\x}) -- (0.01,{0.1*\x});
	\node at (-0.01,{0.1*\x}) [left] {0.\x\strut};
}
\draw (1,-0.01) -- (1,0.01);
\node at (1,-0.01) [below] {1.0\strut};
\node at (0,-0.01) [below] {0\strut};

\draw (-0.01,1) -- (0.01,1);
\node at (-0.01,1) [left] {1.0\strut};

\node[color=darkgreen] at (1,1) [below left,rotate={atan(0.5)}] {$n=0$};
\node[color=darkgreen] at (1,0.5) [above left] {$n=1$};
\node[color=darkgreen] at (1,0.13) [above left,rotate=-13] {$n=2$};
\node[color=darkgreen] at (1,0.00) [above left,rotate=-9] {$n=3$};

\end{scope}

\end{tikzpicture}
\end{document}

