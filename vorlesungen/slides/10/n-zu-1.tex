%
% n-zu-1.tex -- Umwandlend einer DGL n-ter Ordnung in ein System 1. Ordnung
%
% (c) 2021 Prof Dr Andreas Müller, OST Ostschweizer Fachhochschule
% Erstellt durch Roy Seitz
%
% !TeX spellcheck = de_CH
\bgroup
\begin{frame}[t]
  \setlength{\abovedisplayskip}{5pt}
  \setlength{\belowdisplayskip}{5pt}
  %\frametitle{Reicht $1.$ Ordnung?}
  %\vspace{-20pt}
  \begin{columns}[t,onlytextwidth]
    \begin{column}{0.48\textwidth}
      \uncover<1->{
        \begin{block}{Beispiel: DGL 3.~Ordnung} \vspace*{-1ex}
          \begin{align*}
            x^{(3)} + a_2 \ddot x + a_1 \dot x + a_0 x = 0 \\
            \Rightarrow
            x^{(3)} = -a_2 \ddot x - a_1 \dot x - a_0 x
          \end{align*}
        \end{block}
      }
      \uncover<2->{
        \begin{block}{Ziel: Nur noch 1.~Ableitungen}
          Einführen neuer Variablen:
          \begin{align*}
            x_0 &\coloneqq x &
            x_1 &\coloneqq \dot  x &
            x_2 &\coloneqq \ddot x
          \end{align*}
          System von Gleichungen 1.~Ordnung
          \begin{align*}
            \dot x_0 &= x_1 \\
            \dot x_1 &= x_2 \\
            \dot x_2 &= -a_2 x_2 - a_1 x_1 - a_0 x_0
          \end{align*}
        \end{block}
      }
    \end{column}
    \uncover<3->{
      \begin{column}{0.48\textwidth}
        \begin{block}{Als Vektor-Gleichung} \vspace*{-1ex}
          \begin{align*}
            \frac{d}{dt}
            \begin{pmatrix} x_0 \\ x_1 \\ x_2 \end{pmatrix}
            = \begin{pmatrix}
              0     & 1     & 0   \\
              0     & 0     & 1   \\
              -a_0  & -a_1  & -a_2 
            \end{pmatrix}
            \begin{pmatrix} x_0 \\ x_1 \\ x_2 \end{pmatrix}    
          \end{align*}
          
          \uncover<4->{Geht für jede lineare Differentialgleichung!}
          
        \end{block}
      \end{column}
    }
  \end{columns}
\end{frame}
\egroup
