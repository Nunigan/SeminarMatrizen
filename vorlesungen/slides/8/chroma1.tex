%
% chroma1.tex -- slide template
%
% (c) 2021 Prof Dr Andreas Müller, OST Ostschweizer Fachhochschule
%
\bgroup
\begin{frame}[t]
\setlength{\abovedisplayskip}{5pt}
\setlength{\belowdisplayskip}{5pt}
\frametitle{Schranke für $\operatorname{chr}(G)$}
\vspace{-20pt}
\begin{columns}[t,onlytextwidth]
\begin{column}{0.40\textwidth}
\begin{block}{Proposition}
Ist $G$ ein Graph mit maximalem Grad $d$, dann gilt
\[
\operatorname{chr}(G) \le d + 1
\]
\end{block}
\uncover<2->{%
\begin{block}{Beispiel}
\begin{itemize}
\item<3->
Peterson-Graph $G$: maximaler Grad ist $d=3$, aber
\[
\operatorname{chr}(G)
=
3
< d+1=4
\]
\item<4->
Voller Graph $V$: maximaler Grad ist $d=n-1$,
\[
\operatorname{chr}(V) = n = d+1
\]
\end{itemize}
\end{block}}
\end{column}
\begin{column}{0.58\textwidth}
\uncover<4->{%
\begin{proof}[Beweis]
Mit vollständiger Induktion, d.~h.~Annahme: Graphen mit $<n$ Knoten und
maximalem Grad $d$ lassen sich mit höchstens $d+1$ Farben färben.
\begin{itemize}
\item<5-> $X$ ein Graph mit $n$ Knoten
\item<6-> entferne den Knoten $v\in X$, $X'=X\setminus\{v\}$
\item<7-> $X'$ lässt sich mit höchstens $d+1$ Farben einfärben
\item<8-> $v$ hat höchstens $d$ Nachbarn, die höchsten $d$ verschiedene
Farben haben
\item<9-> Es bleibt eine Farbe für $v$
\end{itemize}
\end{proof}}
\end{column}
\end{columns}
\end{frame}
\egroup
