%
% algebrastruktur.tex
%
% (c) 2021 Prof Dr Andreas Müller, OST Ostschweizer Fachhochschule
%
\bgroup

\definecolor{darkgreen}{rgb}{0,0.6,0}

\begin{frame}[t]
\frametitle{Algebra über $\Bbbk$}
\begin{center}
\begin{tikzpicture}[>=latex,thick]
\pgfmathparse{atan(7/4)}
\xdef\a{\pgfmathresult}
\uncover<2->{
	\fill[color=red!40,opacity=0.5]
		({-4-2.5},{2+1.0})
		--
		({-2.5},{-3-1.0})
		--
		({2.5},{-3-1.0})
		--
		({-4+2.5},{2+1.0})
		-- cycle;
}

\uncover<4->{
	\fill[color=blue!40,opacity=0.5]
		({4-2.5},{2+1.0})
		--
		({-2.5},{-3-1.0})
		--
		({2.5},{-3-1.0})
		--
		({4+2.5},{2+1.0})
		-- cycle;
}

\uncover<6->{
	\fill[color=darkgreen!40,opacity=0.5]
		({-4-2.5},{2+1.0})
		-- 
		({-4-2.5+2*(4/7)},{2-1})
		-- 
		({+4+2.5-2*(4/7)},{2-1})
		-- 
		({+4+2.5},{2+1})
		--
		cycle;
}

\node at ({-3-0.5},2) {Skalarmultiplikation};

\node at (3.5,2.2) {Multiplikation};
\node at (3.5,1.8) {\tiny Monoid};

\node at (0,-2.8) {Addition};
\node at (0,-3.2) {\tiny Gruppe};

\uncover<4->{
	\node[color=blue] at (4.8,-0.5) [rotate=\a] {Ring\strut};
}

\uncover<2->{
	\node[color=red] at (-4.8,-0.5) [rotate=-\a] {Vektorraum\strut};
}

\uncover<6->{
	\node[color=darkgreen] at (0,2.6) {$(\lambda a)b=\lambda(ab)$};
}

\uncover<3->{
	\node[color=red] at (-2.5,-0.5) {$\displaystyle
	\begin{aligned}
	\lambda(a+b)&=\lambda a + \lambda b\\
	(\lambda+\mu)a&=\lambda a +\mu a
	\end{aligned}$};
}

\uncover<5->{
	\node[color=blue] at (2.5,-0.5) {$\displaystyle
	\begin{aligned}
	a(b+c)&=ab+ac\\
	(a+b)c&=ac+bc
	\end{aligned}$};
}

\end{tikzpicture}
\end{center}
\end{frame}

\egroup
