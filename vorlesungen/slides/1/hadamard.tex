%
% hadamard.tex
%
% (c) 2021 Prof Dr Andreas Müller, OST Ostschweizer Fachhochschule
%
\begin{frame}[t]
\frametitle{Hadamard-Algebra}
\begin{block}{Alternatives Produkt: Hadamard-Produkt}
\[
\begin{pmatrix}
a_{11}&\dots&a_{1n}\\
\vdots&\ddots&\vdots\\
a_{m1}&\dots&a_{mn}\\
\end{pmatrix}
\odot
\begin{pmatrix}
b_{11}&\dots&b_{1n}\\
\vdots&\ddots&\vdots\\
b_{m1}&\dots&b_{mn}\\
\end{pmatrix}
=
\begin{pmatrix}
a_{11}b_{11}&\dots&a_{1n}b_{1n}\\
\vdots&\ddots&\vdots\\
a_{m1}b_{m1}&\dots&a_{mn}b_{mn}\\
\end{pmatrix}
\]
\end{block}
\vspace{-10pt}
\begin{columns}[t,onlytextwidth]
\begin{column}{0.58\textwidth}
\uncover<2->{%
\begin{block}{Algebra}
\begin{itemize}
\item<3-> $M_{mn}(\Bbbk)$ ist eine Algebra mit
$\odot$ als Produkt
\item<4-> Neutrales Element $U$: Matrix aus lauter Einsen
\item<5-> Anwendung: Wahrscheinlichkeitsmatrizen
\end{itemize}
\end{block}}
\end{column}
\begin{column}{0.38\textwidth}
\uncover<6->{%
\begin{block}{Nicht so interessant}
Die Hadamard-Algebra ist kommutativ 
\uncover<7->{$\Rightarrow$ 
kann ``keine'' interessanten algebraischen Relationen darstellen}
\end{block}}
\end{column}
\end{columns}
\end{frame}
