%
% ganz.tex
%
% (c) 2021 Prof Dr Andreas Müller, Hochschule Rapperswil
%
\begin{frame}[t]
\frametitle{Ganze Zahlen: Gruppe}
\vspace{-20pt}
\begin{columns}[t,onlytextwidth]
\begin{column}{0.48\textwidth}
\setlength{\abovedisplayskip}{5pt}
\setlength{\belowdisplayskip}{5pt}
\begin{block}{Subtrahieren}
Nicht für alle $a,b\in \mathbb{N}$ hat die 
Gleichung
\[
a+x=b
\uncover<2->{
\quad
\Rightarrow
\quad
x=b-a}
\]
eine Lösung in $\mathbb{N}$\uncover<2->{, nämlich wenn $a>b$}%
\end{block}
\uncover<3->{%
\begin{block}{Ganze Zahlen = Paare}
Idee: $b-a = (b,a)$
\begin{enumerate}
\item<4-> $(b,a)=\mathbb{N}\times\mathbb{N}$
\item<5-> Äquivalenzrelation
\[
(b,a)\sim (d,c)
\ifthenelse{\boolean{presentation}}{
\only<6>{\Leftrightarrow
\text{``\strut}
b-a=c-d
\text{\strut''}}}{}
\only<7->{
\Leftrightarrow
b+d=c+a}
\]
\end{enumerate}
\vspace{-10pt}
\uncover<8->{%
Ganze Zahlen:
\(
\mathbb{Z}
=
\mathbb{N}\times\mathbb{N}/\sim
\)}
\\
\uncover<9->{%
$z\in\mathbb{Z}$, $z=\mathstrut$ Paare $(u,v)$ mit 
``gleicher Differenz''}
\uncover<10->{%
$\Rightarrow$ alle Differenzen in $\mathbb{Z}$}
\end{block}}
\end{column}
\begin{column}{0.48\textwidth}
\setlength{\abovedisplayskip}{5pt}
\setlength{\belowdisplayskip}{5pt}
\uncover<11->{%
\begin{block}{Gruppe}
Monoid $\ifthenelse{\boolean{presentation}}{\only<11>{\mathbb{Z}}}{}\only<12->{G}$ mit inversem Element
\[
a\in \ifthenelse{\boolean{presentation}}{\only<11>{\mathbb{Z}}}{}\only<12->{G}
\Rightarrow
\ifthenelse{\boolean{presentation}}{\only<11>{-a\in\mathbb{Z}}}{}\only<12->{a^{-1}\in G}
\text{ mit }
\ifthenelse{\boolean{presentation}}{
\only<11>{
a+(-a)=0
}}{}
\only<12->{
\left\{
\begin{aligned}
aa^{-1}&=e
\\
a^{-1}a&=e
\end{aligned}
\right.
}
\]
\end{block}}
\vspace{-15pt}
\uncover<13->{%
\begin{block}{Abelsche Gruppe}
Verknüpfung ist kommutativ:
\[
a+b=b+a
\]
\end{block}}
\vspace{-12pt}
\uncover<14->{%
\begin{block}{Beispiele}
\begin{itemize}
\item<15-> Brüche, reelle Zahlen
\item<16-> invertierbare Matrizen: $\operatorname{GL}_n(\mathbb{R})$
\item<17-> Drehmatrizen: $\operatorname{SO}(n)$
\item<18-> Matrizen mit Determinante $1$: $\operatorname{SL}_n(\mathbb R)$
\end{itemize}
\end{block}}
\end{column}
\end{columns}
\end{frame}
