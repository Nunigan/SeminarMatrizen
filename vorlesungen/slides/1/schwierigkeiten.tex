%
% schwierigkeiten.tex
%
% (c) 2021 Prof Dr Andreas Müller, Hochschule Rapperswil
%
\begin{frame}[t]
\frametitle{Schwierigkeiten}
\vspace{-15pt}
\setlength{\abovedisplayskip}{5pt}
\setlength{\belowdisplayskip}{5pt}
\begin{columns}[t,onlytextwidth]
\begin{column}{0.48\textwidth}
\uncover<2->{%
\begin{block}{Nullteiler}
Elemente $a,b$ mit $ab=0$
$\Rightarrow$ nicht invertierbar
\begin{itemize}
\item<3-> Projektionen
\[
\begin{pmatrix}
1&0\\0&0
\end{pmatrix}
\begin{pmatrix}
0&0\\0&1
\end{pmatrix}
=
0
\]
\item<4-> Nilpotente Matrizen
\[
\begin{pmatrix}
0&1&0\\
0&0&1\\
0&0&0
\end{pmatrix}^3
=0
\]
\item<5->
In $\mathbb{Z}/15\mathbb{Z}$ (modulo 15):
\[
3\cdot 5 = 15 \equiv 0\mod 15
\]
\end{itemize}
\end{block}}
\end{column}
\begin{column}{0.48\textwidth}
\uncover<6->{%
\begin{block}{Invertierbarkeit}
\begin{itemize}
\item<7->
$7\in\mathbb{Z}$, aber $7^{-1}\not\in\mathbb{Z}$, $7^{-1}\in\mathbb{Q}$
\item<8->
$A$ regulär heisst nicht $A^{-1}\in M_n(\mathbb{Z})$
\[
A=\begin{pmatrix}
1&-1\\
1&1
\end{pmatrix}
\;\Rightarrow\;
A^{-1}
=
\begin{pmatrix}
\frac12&\frac12\\
-\frac12&\frac12
\end{pmatrix}
\]
\item<9->
$A\in\operatorname{SL}_n(\mathbb{Z})$ invertierbar in
$M_n(\mathbb{Z})$:
\[
A=
\begin{pmatrix}
5&4\\4&3
\end{pmatrix}
\;
\Rightarrow
\;
A^{-1}=
\begin{pmatrix}
-3&4\\4&-5
\end{pmatrix}
\]
\end{itemize}
\uncover<10->{%
Invertierbarkeit erreichen durch ``vergrössern'' des Ringes
}
\end{block}}
\end{column}
\end{columns}
\end{frame}
