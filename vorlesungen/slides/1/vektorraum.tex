%
% vektorraum.tex
%
% (c) 2021 Prof Dr Andreas Müller, OST Ostschweizer Fachhochschule
%
\begin{frame}[t]
\frametitle{Vektorraum}
\vspace{-10pt}
\begin{columns}[t,onlytextwidth]
\begin{column}{0.48\textwidth}
\begin{block}{Operationen}
Addition:
\[
\begin{pmatrix}a_1\\\vdots\\a_n \end{pmatrix}
+
\begin{pmatrix}b_1\\\vdots\\b_n \end{pmatrix}
=
\begin{pmatrix}a_1+b_1\\\vdots\\a_n+b_n \end{pmatrix}
\]
Skalarmultiplikation:
\[
\lambda\begin{pmatrix}a_1\\\vdots\\a_n \end{pmatrix}
=
\begin{pmatrix}\lambda a_1\\\vdots\\\lambda a_n \end{pmatrix}
\]
\end{block}
\end{column}
\begin{column}{0.48\textwidth}
\uncover<2->{%
\begin{block}{Additive Gruppe}
$\mathbb{R}^n$ ist eine Gruppe bezüglich der Addition
mit 
\[
0=\begin{pmatrix}0\\\vdots\\0\end{pmatrix},
\qquad
-a
=
-\begin{pmatrix}a_1\\\vdots\\a_n\end{pmatrix}
=
\begin{pmatrix}-a_1\\\vdots\\-a_n\end{pmatrix}
\]
\end{block}}
\vspace{-5pt}
\uncover<3->{%
\begin{block}{Skalarmultiplikation}
Distributivgesetz
\begin{align*}
(\lambda+\mu)a&=\lambda a + \mu a\\
\lambda (a+b)&=\lambda a + \lambda b
\end{align*}
\end{block}}
\end{column}
\end{columns}
\end{frame}
