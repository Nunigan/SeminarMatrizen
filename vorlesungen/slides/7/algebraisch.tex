%
% algebraisch.tex -- algebraische Definition der Symmetrien
%
% (c) 2021 Prof Dr Andreas Müller, OST Ostschweizer Fachhochschule
%
\bgroup
\begin{frame}[t]
\setlength{\abovedisplayskip}{5pt}
\setlength{\belowdisplayskip}{5pt}
\frametitle{Erhaltungsgrössen und Algebra}
\vspace{-20pt}
\begin{columns}[t,onlytextwidth]
\begin{column}{0.48\textwidth}
\begin{block}{Längen und Winkel}
Längenmessung mit Skalarprodukt
\begin{align*}
\|\vec{v}\|^2
&=
\langle \vec{v},\vec{v}\rangle
=
\vec{v}\cdot \vec{v}
\uncover<2->{=
\vec{v}^t\vec{v}}
\end{align*}
\end{block}
\end{column}
\begin{column}{0.48\textwidth}
\uncover<3->{%
\begin{block}{Flächeninhalt/Volumen}
$n$ Vektoren $V=(\vec{v}_1,\dots,\vec{v}_n)$
\\
Volumen des Parallelepipeds: $\det V$
\end{block}}
\end{column}
\end{columns}
%
\vspace{-7pt}
\begin{columns}[t,onlytextwidth]
\begin{column}{0.48\textwidth}
\uncover<4->{%
\begin{block}{Längenerhaltende Transformationen}
$A\in\operatorname{GL}_n(\mathbb{R})$
\begin{align*}
\vec{x}^t\vec{y}
&=
(A\vec{x})
\cdot
(A\vec{y})
\uncover<5->{=
(A\vec{x})^t
(A\vec{y})}
\\
\uncover<6->{
\vec{x}^tI\vec{y}
&=
\vec{x}^tA^tA\vec{y}}
\uncover<7->{
\Rightarrow I=A^tA}
\end{align*}
\uncover<8->{Begründung: $\vec{e}_i^t B \vec{e}_j = b_{ij}$}
\end{block}}
\end{column}
\begin{column}{0.48\textwidth}
\uncover<9->{%
\begin{block}{Volumenerhaltende Transformationen}
$A\in\operatorname{GL}_n(\mathbb{R})$
\begin{align*}
\det(V)
&=
\det(AV)
\uncover<10->{=
\det(A)\det(V)}
\\
\uncover<11->{
1&=\det(A)}
\end{align*}
\uncover<10->{
(Produktsatz für Determinante)
}
\end{block}}
\end{column}
\end{columns}
%
\vspace{-3pt}
\begin{columns}[t,onlytextwidth]
\begin{column}{0.48\textwidth}
\uncover<12->{%
\begin{block}{Orthogonale Matrizen}
Längentreue Abbildungen = orthogonale Matrizen:
\[
O(n)
=
\{
A \in \operatorname{GL}_n(\mathbb{R})
\;|\;
A^tA=I
\}
\]
\end{block}}
\end{column}
\begin{column}{0.48\textwidth}
\uncover<13->{%
\begin{block}{``Spezielle'' Matrizen}
Volumen-/Orientierungserhaltende Transformationen:
\[
\operatorname{SL}_n(\mathbb R) 
=
\{ A \in \operatorname{GL}_n(\mathbb{R}) \;|\; \det A = 1\}
\]
\end{block}}
\end{column}
\end{columns}

\end{frame}
\egroup
