%
% slides.tex collection of all slides
%
% (c) 2019 Prof Dr Andreas Müller, Hochschule Rapperswil
%
\def\titel{
% title slide for this chapter
\begin{frame}
\titlepage
\end{frame}
\ifthenelse{\boolean{presentation}}{}{
% add an empty slide for alignment in the slide catalog
\begin{frame}
\end{frame}
}
}

\title[MathSem]{Mathematisches Seminar: Matrizen}
\section{Intro}
\titel
\author[]{}
\subtitle{}
%
% chapter.tex
%
% (c) 2021 Prof Dr Andreas Müller, Hochschule Rapperswi
%
\folie{5/plan.tex}
\folie{5/planbeispiele.tex}
\folie{5/verzerrung.tex}
\folie{5/motivation.tex}
\folie{5/charpoly.tex}
\folie{5/kernbildintro.tex}
\folie{5/kernbilder.tex}
\folie{5/kernbild.tex}
\folie{5/ketten.tex}
\folie{5/dimension.tex}
\folie{5/folgerungen.tex}
\folie{5/injektiv.tex}
\folie{5/nilpotent.tex}
\folie{5/eigenraeume.tex}
\folie{5/zerlegung.tex}
\folie{5/normalnilp.tex}
\folie{5/bloecke.tex}
\folie{5/jordanblock.tex}
\folie{5/jordan.tex}
\folie{5/reellenormalform.tex}
\folie{5/cayleyhamilton.tex}
\folie{5/konvergenzradius.tex}
\folie{5/krbeispiele.tex}
\folie{5/spektralgelfand.tex}
\folie{5/Aiteration.tex}
\folie{5/satzvongelfand.tex}
\folie{5/stoneweierstrass.tex}
\folie{5/swbeweis.tex}
\folie{5/potenzreihenmethode.tex}
\folie{5/logarithmusreihe.tex}
\folie{5/exponentialfunktion.tex}
\folie{5/hyperbolisch.tex}
\folie{5/spektrum.tex}
\folie{5/normal.tex}
\folie{5/normalbeispiel.tex}
\folie{5/normalbeispiel34.tex}
\folie{5/approximation.tex}


\title[Grundlagen]{Grundlagen}
\section{Grundlagen}
\titel
%
% chapter.tex
%
% (c) 2021 Prof Dr Andreas Müller, Hochschule Rapperswi
%
\folie{5/plan.tex}
\folie{5/planbeispiele.tex}
\folie{5/verzerrung.tex}
\folie{5/motivation.tex}
\folie{5/charpoly.tex}
\folie{5/kernbildintro.tex}
\folie{5/kernbilder.tex}
\folie{5/kernbild.tex}
\folie{5/ketten.tex}
\folie{5/dimension.tex}
\folie{5/folgerungen.tex}
\folie{5/injektiv.tex}
\folie{5/nilpotent.tex}
\folie{5/eigenraeume.tex}
\folie{5/zerlegung.tex}
\folie{5/normalnilp.tex}
\folie{5/bloecke.tex}
\folie{5/jordanblock.tex}
\folie{5/jordan.tex}
\folie{5/reellenormalform.tex}
\folie{5/cayleyhamilton.tex}
\folie{5/konvergenzradius.tex}
\folie{5/krbeispiele.tex}
\folie{5/spektralgelfand.tex}
\folie{5/Aiteration.tex}
\folie{5/satzvongelfand.tex}
\folie{5/stoneweierstrass.tex}
\folie{5/swbeweis.tex}
\folie{5/potenzreihenmethode.tex}
\folie{5/logarithmusreihe.tex}
\folie{5/exponentialfunktion.tex}
\folie{5/hyperbolisch.tex}
\folie{5/spektrum.tex}
\folie{5/normal.tex}
\folie{5/normalbeispiel.tex}
\folie{5/normalbeispiel34.tex}
\folie{5/approximation.tex}


\title[Vektoren/Matrizen]{Vektoren und Matrizen}
\section{Vektoren und Matrizen}
\titel
%
% chapter.tex
%
% (c) 2021 Prof Dr Andreas Müller, Hochschule Rapperswi
%
\folie{5/plan.tex}
\folie{5/planbeispiele.tex}
\folie{5/verzerrung.tex}
\folie{5/motivation.tex}
\folie{5/charpoly.tex}
\folie{5/kernbildintro.tex}
\folie{5/kernbilder.tex}
\folie{5/kernbild.tex}
\folie{5/ketten.tex}
\folie{5/dimension.tex}
\folie{5/folgerungen.tex}
\folie{5/injektiv.tex}
\folie{5/nilpotent.tex}
\folie{5/eigenraeume.tex}
\folie{5/zerlegung.tex}
\folie{5/normalnilp.tex}
\folie{5/bloecke.tex}
\folie{5/jordanblock.tex}
\folie{5/jordan.tex}
\folie{5/reellenormalform.tex}
\folie{5/cayleyhamilton.tex}
\folie{5/konvergenzradius.tex}
\folie{5/krbeispiele.tex}
\folie{5/spektralgelfand.tex}
\folie{5/Aiteration.tex}
\folie{5/satzvongelfand.tex}
\folie{5/stoneweierstrass.tex}
\folie{5/swbeweis.tex}
\folie{5/potenzreihenmethode.tex}
\folie{5/logarithmusreihe.tex}
\folie{5/exponentialfunktion.tex}
\folie{5/hyperbolisch.tex}
\folie{5/spektrum.tex}
\folie{5/normal.tex}
\folie{5/normalbeispiel.tex}
\folie{5/normalbeispiel34.tex}
\folie{5/approximation.tex}


\title[Polynome]{Polynome}
\section{Polynome}
\titel
%
% chapter.tex
%
% (c) 2021 Prof Dr Andreas Müller, Hochschule Rapperswi
%
\folie{5/plan.tex}
\folie{5/planbeispiele.tex}
\folie{5/verzerrung.tex}
\folie{5/motivation.tex}
\folie{5/charpoly.tex}
\folie{5/kernbildintro.tex}
\folie{5/kernbilder.tex}
\folie{5/kernbild.tex}
\folie{5/ketten.tex}
\folie{5/dimension.tex}
\folie{5/folgerungen.tex}
\folie{5/injektiv.tex}
\folie{5/nilpotent.tex}
\folie{5/eigenraeume.tex}
\folie{5/zerlegung.tex}
\folie{5/normalnilp.tex}
\folie{5/bloecke.tex}
\folie{5/jordanblock.tex}
\folie{5/jordan.tex}
\folie{5/reellenormalform.tex}
\folie{5/cayleyhamilton.tex}
\folie{5/konvergenzradius.tex}
\folie{5/krbeispiele.tex}
\folie{5/spektralgelfand.tex}
\folie{5/Aiteration.tex}
\folie{5/satzvongelfand.tex}
\folie{5/stoneweierstrass.tex}
\folie{5/swbeweis.tex}
\folie{5/potenzreihenmethode.tex}
\folie{5/logarithmusreihe.tex}
\folie{5/exponentialfunktion.tex}
\folie{5/hyperbolisch.tex}
\folie{5/spektrum.tex}
\folie{5/normal.tex}
\folie{5/normalbeispiel.tex}
\folie{5/normalbeispiel34.tex}
\folie{5/approximation.tex}


\title[Endliche Körper]{Endliche Körper}
\section{Endliche Körper}
\titel
%
% chapter.tex
%
% (c) 2021 Prof Dr Andreas Müller, Hochschule Rapperswi
%
\folie{5/plan.tex}
\folie{5/planbeispiele.tex}
\folie{5/verzerrung.tex}
\folie{5/motivation.tex}
\folie{5/charpoly.tex}
\folie{5/kernbildintro.tex}
\folie{5/kernbilder.tex}
\folie{5/kernbild.tex}
\folie{5/ketten.tex}
\folie{5/dimension.tex}
\folie{5/folgerungen.tex}
\folie{5/injektiv.tex}
\folie{5/nilpotent.tex}
\folie{5/eigenraeume.tex}
\folie{5/zerlegung.tex}
\folie{5/normalnilp.tex}
\folie{5/bloecke.tex}
\folie{5/jordanblock.tex}
\folie{5/jordan.tex}
\folie{5/reellenormalform.tex}
\folie{5/cayleyhamilton.tex}
\folie{5/konvergenzradius.tex}
\folie{5/krbeispiele.tex}
\folie{5/spektralgelfand.tex}
\folie{5/Aiteration.tex}
\folie{5/satzvongelfand.tex}
\folie{5/stoneweierstrass.tex}
\folie{5/swbeweis.tex}
\folie{5/potenzreihenmethode.tex}
\folie{5/logarithmusreihe.tex}
\folie{5/exponentialfunktion.tex}
\folie{5/hyperbolisch.tex}
\folie{5/spektrum.tex}
\folie{5/normal.tex}
\folie{5/normalbeispiel.tex}
\folie{5/normalbeispiel34.tex}
\folie{5/approximation.tex}


\title[EW/EV]{Eigenwerte und Eigenvektoren}
\section{Eigenwerte und Eigenvektoren}
\titel
%
% chapter.tex
%
% (c) 2021 Prof Dr Andreas Müller, Hochschule Rapperswi
%
\folie{5/plan.tex}
\folie{5/planbeispiele.tex}
\folie{5/verzerrung.tex}
\folie{5/motivation.tex}
\folie{5/charpoly.tex}
\folie{5/kernbildintro.tex}
\folie{5/kernbilder.tex}
\folie{5/kernbild.tex}
\folie{5/ketten.tex}
\folie{5/dimension.tex}
\folie{5/folgerungen.tex}
\folie{5/injektiv.tex}
\folie{5/nilpotent.tex}
\folie{5/eigenraeume.tex}
\folie{5/zerlegung.tex}
\folie{5/normalnilp.tex}
\folie{5/bloecke.tex}
\folie{5/jordanblock.tex}
\folie{5/jordan.tex}
\folie{5/reellenormalform.tex}
\folie{5/cayleyhamilton.tex}
\folie{5/konvergenzradius.tex}
\folie{5/krbeispiele.tex}
\folie{5/spektralgelfand.tex}
\folie{5/Aiteration.tex}
\folie{5/satzvongelfand.tex}
\folie{5/stoneweierstrass.tex}
\folie{5/swbeweis.tex}
\folie{5/potenzreihenmethode.tex}
\folie{5/logarithmusreihe.tex}
\folie{5/exponentialfunktion.tex}
\folie{5/hyperbolisch.tex}
\folie{5/spektrum.tex}
\folie{5/normal.tex}
\folie{5/normalbeispiel.tex}
\folie{5/normalbeispiel34.tex}
\folie{5/approximation.tex}


%\title[Permutationen]{Permutationen}
%\section{Permutationen}
%\titel
%%
% chapter.tex
%
% (c) 2021 Prof Dr Andreas Müller, Hochschule Rapperswi
%
\folie{5/plan.tex}
\folie{5/planbeispiele.tex}
\folie{5/verzerrung.tex}
\folie{5/motivation.tex}
\folie{5/charpoly.tex}
\folie{5/kernbildintro.tex}
\folie{5/kernbilder.tex}
\folie{5/kernbild.tex}
\folie{5/ketten.tex}
\folie{5/dimension.tex}
\folie{5/folgerungen.tex}
\folie{5/injektiv.tex}
\folie{5/nilpotent.tex}
\folie{5/eigenraeume.tex}
\folie{5/zerlegung.tex}
\folie{5/normalnilp.tex}
\folie{5/bloecke.tex}
\folie{5/jordanblock.tex}
\folie{5/jordan.tex}
\folie{5/reellenormalform.tex}
\folie{5/cayleyhamilton.tex}
\folie{5/konvergenzradius.tex}
\folie{5/krbeispiele.tex}
\folie{5/spektralgelfand.tex}
\folie{5/Aiteration.tex}
\folie{5/satzvongelfand.tex}
\folie{5/stoneweierstrass.tex}
\folie{5/swbeweis.tex}
\folie{5/potenzreihenmethode.tex}
\folie{5/logarithmusreihe.tex}
\folie{5/exponentialfunktion.tex}
\folie{5/hyperbolisch.tex}
\folie{5/spektrum.tex}
\folie{5/normal.tex}
\folie{5/normalbeispiel.tex}
\folie{5/normalbeispiel34.tex}
\folie{5/approximation.tex}


%\title[Matrizengruppen]{Matrizengruppen}
%\section{Matrizengruppen}
%\titel
%%
% chapter.tex
%
% (c) 2021 Prof Dr Andreas Müller, Hochschule Rapperswi
%
\folie{5/plan.tex}
\folie{5/planbeispiele.tex}
\folie{5/verzerrung.tex}
\folie{5/motivation.tex}
\folie{5/charpoly.tex}
\folie{5/kernbildintro.tex}
\folie{5/kernbilder.tex}
\folie{5/kernbild.tex}
\folie{5/ketten.tex}
\folie{5/dimension.tex}
\folie{5/folgerungen.tex}
\folie{5/injektiv.tex}
\folie{5/nilpotent.tex}
\folie{5/eigenraeume.tex}
\folie{5/zerlegung.tex}
\folie{5/normalnilp.tex}
\folie{5/bloecke.tex}
\folie{5/jordanblock.tex}
\folie{5/jordan.tex}
\folie{5/reellenormalform.tex}
\folie{5/cayleyhamilton.tex}
\folie{5/konvergenzradius.tex}
\folie{5/krbeispiele.tex}
\folie{5/spektralgelfand.tex}
\folie{5/Aiteration.tex}
\folie{5/satzvongelfand.tex}
\folie{5/stoneweierstrass.tex}
\folie{5/swbeweis.tex}
\folie{5/potenzreihenmethode.tex}
\folie{5/logarithmusreihe.tex}
\folie{5/exponentialfunktion.tex}
\folie{5/hyperbolisch.tex}
\folie{5/spektrum.tex}
\folie{5/normal.tex}
\folie{5/normalbeispiel.tex}
\folie{5/normalbeispiel34.tex}
\folie{5/approximation.tex}


\title[Graphen]{Graphen}
\section{Graphen}
\titel
%
% chapter.tex
%
% (c) 2021 Prof Dr Andreas Müller, Hochschule Rapperswi
%
\folie{5/plan.tex}
\folie{5/planbeispiele.tex}
\folie{5/verzerrung.tex}
\folie{5/motivation.tex}
\folie{5/charpoly.tex}
\folie{5/kernbildintro.tex}
\folie{5/kernbilder.tex}
\folie{5/kernbild.tex}
\folie{5/ketten.tex}
\folie{5/dimension.tex}
\folie{5/folgerungen.tex}
\folie{5/injektiv.tex}
\folie{5/nilpotent.tex}
\folie{5/eigenraeume.tex}
\folie{5/zerlegung.tex}
\folie{5/normalnilp.tex}
\folie{5/bloecke.tex}
\folie{5/jordanblock.tex}
\folie{5/jordan.tex}
\folie{5/reellenormalform.tex}
\folie{5/cayleyhamilton.tex}
\folie{5/konvergenzradius.tex}
\folie{5/krbeispiele.tex}
\folie{5/spektralgelfand.tex}
\folie{5/Aiteration.tex}
\folie{5/satzvongelfand.tex}
\folie{5/stoneweierstrass.tex}
\folie{5/swbeweis.tex}
\folie{5/potenzreihenmethode.tex}
\folie{5/logarithmusreihe.tex}
\folie{5/exponentialfunktion.tex}
\folie{5/hyperbolisch.tex}
\folie{5/spektrum.tex}
\folie{5/normal.tex}
\folie{5/normalbeispiel.tex}
\folie{5/normalbeispiel34.tex}
\folie{5/approximation.tex}


\title[Wahrscheinlichkeit]{Wahrscheinlichkeitsmatrizen}
\section{Wahrscheinlichkeitsmatrizen}
\titel
%
% chapter.tex
%
% (c) 2021 Prof Dr Andreas Müller, Hochschule Rapperswi
%
\folie{5/plan.tex}
\folie{5/planbeispiele.tex}
\folie{5/verzerrung.tex}
\folie{5/motivation.tex}
\folie{5/charpoly.tex}
\folie{5/kernbildintro.tex}
\folie{5/kernbilder.tex}
\folie{5/kernbild.tex}
\folie{5/ketten.tex}
\folie{5/dimension.tex}
\folie{5/folgerungen.tex}
\folie{5/injektiv.tex}
\folie{5/nilpotent.tex}
\folie{5/eigenraeume.tex}
\folie{5/zerlegung.tex}
\folie{5/normalnilp.tex}
\folie{5/bloecke.tex}
\folie{5/jordanblock.tex}
\folie{5/jordan.tex}
\folie{5/reellenormalform.tex}
\folie{5/cayleyhamilton.tex}
\folie{5/konvergenzradius.tex}
\folie{5/krbeispiele.tex}
\folie{5/spektralgelfand.tex}
\folie{5/Aiteration.tex}
\folie{5/satzvongelfand.tex}
\folie{5/stoneweierstrass.tex}
\folie{5/swbeweis.tex}
\folie{5/potenzreihenmethode.tex}
\folie{5/logarithmusreihe.tex}
\folie{5/exponentialfunktion.tex}
\folie{5/hyperbolisch.tex}
\folie{5/spektrum.tex}
\folie{5/normal.tex}
\folie{5/normalbeispiel.tex}
\folie{5/normalbeispiel34.tex}
\folie{5/approximation.tex}


%\title[Krypto]{Anwendungen in Kryptographie und Codierungstheorie}
%\section{Krypto}
%\titel
%%
% chapter.tex
%
% (c) 2021 Prof Dr Andreas Müller, Hochschule Rapperswi
%
\folie{5/plan.tex}
\folie{5/planbeispiele.tex}
\folie{5/verzerrung.tex}
\folie{5/motivation.tex}
\folie{5/charpoly.tex}
\folie{5/kernbildintro.tex}
\folie{5/kernbilder.tex}
\folie{5/kernbild.tex}
\folie{5/ketten.tex}
\folie{5/dimension.tex}
\folie{5/folgerungen.tex}
\folie{5/injektiv.tex}
\folie{5/nilpotent.tex}
\folie{5/eigenraeume.tex}
\folie{5/zerlegung.tex}
\folie{5/normalnilp.tex}
\folie{5/bloecke.tex}
\folie{5/jordanblock.tex}
\folie{5/jordan.tex}
\folie{5/reellenormalform.tex}
\folie{5/cayleyhamilton.tex}
\folie{5/konvergenzradius.tex}
\folie{5/krbeispiele.tex}
\folie{5/spektralgelfand.tex}
\folie{5/Aiteration.tex}
\folie{5/satzvongelfand.tex}
\folie{5/stoneweierstrass.tex}
\folie{5/swbeweis.tex}
\folie{5/potenzreihenmethode.tex}
\folie{5/logarithmusreihe.tex}
\folie{5/exponentialfunktion.tex}
\folie{5/hyperbolisch.tex}
\folie{5/spektrum.tex}
\folie{5/normal.tex}
\folie{5/normalbeispiel.tex}
\folie{5/normalbeispiel34.tex}
\folie{5/approximation.tex}


%\title[Homologie]{Homologie}
%\section{Homologie}
%\titel
%%
% chapter.tex
%
% (c) 2021 Prof Dr Andreas Müller, Hochschule Rapperswi
%
\folie{5/plan.tex}
\folie{5/planbeispiele.tex}
\folie{5/verzerrung.tex}
\folie{5/motivation.tex}
\folie{5/charpoly.tex}
\folie{5/kernbildintro.tex}
\folie{5/kernbilder.tex}
\folie{5/kernbild.tex}
\folie{5/ketten.tex}
\folie{5/dimension.tex}
\folie{5/folgerungen.tex}
\folie{5/injektiv.tex}
\folie{5/nilpotent.tex}
\folie{5/eigenraeume.tex}
\folie{5/zerlegung.tex}
\folie{5/normalnilp.tex}
\folie{5/bloecke.tex}
\folie{5/jordanblock.tex}
\folie{5/jordan.tex}
\folie{5/reellenormalform.tex}
\folie{5/cayleyhamilton.tex}
\folie{5/konvergenzradius.tex}
\folie{5/krbeispiele.tex}
\folie{5/spektralgelfand.tex}
\folie{5/Aiteration.tex}
\folie{5/satzvongelfand.tex}
\folie{5/stoneweierstrass.tex}
\folie{5/swbeweis.tex}
\folie{5/potenzreihenmethode.tex}
\folie{5/logarithmusreihe.tex}
\folie{5/exponentialfunktion.tex}
\folie{5/hyperbolisch.tex}
\folie{5/spektrum.tex}
\folie{5/normal.tex}
\folie{5/normalbeispiel.tex}
\folie{5/normalbeispiel34.tex}
\folie{5/approximation.tex}


