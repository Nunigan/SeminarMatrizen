%
% potenzreihenmethode.tex
%
% (c) 2021 Prof Dr Andreas Müller, OST Ostschweizer Fachhochschule
%
\begin{frame}[t]
\setlength{\abovedisplayskip}{5pt}
\setlength{\belowdisplayskip}{5pt}
\frametitle{Potenzreihenmethode}
\vspace{-15pt}
\begin{columns}[t,onlytextwidth]
\begin{column}{0.48\textwidth}
\begin{block}{Lineare Differentialgleichung}
\vspace{-12pt}
\begin{align*}
y'&=ay&&\Rightarrow&y'-ay&=0
\\
y(0)&=C
\end{align*}
\end{block}
\end{column}
\begin{column}{0.48\textwidth}
\uncover<2->{%
\begin{block}{Potenzreihenansatz}
\vspace{-12pt}
\begin{align*}
y(x)
&=
a_0+ a_1x + a_2x^2 + \dots
\\
y(0)&=a_0=C
\end{align*}
\end{block}}
\end{column}
\end{columns}
\uncover<3->{%
\begin{block}{Lösung}
\vspace{-12pt}
\[
\arraycolsep=1.4pt
\begin{array}{rcrcrcrcrcr}
\uncover<3->{ y'(x)}
	\uncover<5->{
	&=&\phantom{(} a_1\phantom{\mathstrut-aa_0)}
	&+& 2a_2\phantom{\mathstrut-aa_1)}x
		&+& 3a_3\phantom{\mathstrut-aa_2)}x^2
			&+& 4a_4\phantom{\mathstrut-aa_3)}x^3
				&+& \dots}\\
\uncover<3->{-ay(x)}
	\uncover<6->{
	&=&\mathstrut-aa_0 \phantom{)}
	&-&  aa_1\phantom{)}x
		&-&  aa_2\phantom{)}x^2
			&-&  aa_3\phantom{)}x^3
				&-& \dots}\\[2pt]
\hline
\\[-10pt]
\uncover<3->{0}
	\uncover<7->{
	&=&(a_1-aa_0)
	&+& (2a_2-aa_1)x
		&+& (3a_3-aa_2)x^2
			&+& (4a_4-aa_3)x^3
				&+& \dots}\\
\end{array}
\]
\begin{align*}
\uncover<4->{
a_0&=C}\uncover<8->{,
\quad
a_1=aa_0=aC}\uncover<9->{,
\quad
a_2=\frac12a^2C}\uncover<10->{,
\quad
a_3=\frac16a^3C}\uncover<11->{,
\dots
a_k=\frac1{k!}a^kC}
\hspace{3cm}
\\
\uncover<4->{
\Rightarrow y(x) &= C}\uncover<8->{+Cax}\uncover<9->{ + C\frac12(ax)^2}
\uncover<10->{ + C \frac16(ax)^3}
\uncover<11->{ + \dots+C\frac{1}{k!}(ax)^k+\dots}
\ifthenelse{\boolean{presentation}}{
\only<12>{
=
C\sum_{k=0}^\infty \frac{(ax)^k}{k!}}
}{}
\uncover<13->{=
Ce^{ax}}
\end{align*}
\end{block}}
\end{frame}
