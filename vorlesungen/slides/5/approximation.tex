%
% approximation.tex
%
% (c) 2021 Prof Dr Andreas Müller, OST Ostschweizer Fachhochschule
%

\begin{frame}[t]
\frametitle{Approximation einer reellen Funktion}
\vspace{-18pt}
\begin{columns}[t,onlytextwidth]
\begin{column}{0.5\textwidth}
\begin{block}{Gegeben}
Eine stetige Funktion $f\colon[a,b]\to\mathbb{R}$
\end{block}
\end{column}
\begin{column}{0.5\textwidth}
\uncover<2->{%
\begin{block}{Gesucht}
Approximationspolynome $p_n\to f$ gleichmässig auf $[a,b]$
\end{block}}
\end{column}
\end{columns}
\uncover<3->{%
\begin{block}{Lösungsmöglichkeiten}
\vspace{-3pt}
\begin{center}
\renewcommand{\arraystretch}{1.3}
\begin{tabular}{|p{4.2cm}|l|}
\hline
Familie&Approximationspolynom für $[a,b]=[0,1]$
\\
\hline
\uncover<4->{%
\raggedright
Lagrange-Interpolationspolynom}
&\uncover<5->{%
$\displaystyle\begin{aligned}
l(x)&=(x-x_0)(x-x_1)\dots(x-x_n),\quad x_k = \frac{k}{n}
\\
p_n(x)&= \sum_{k=0}^n f(x_k)\frac{l(x)}{x-x_k}
\end{aligned}$}
\\
\hline\uncover<6->{%
\raggedright
Approximation mit Bernstein-Polynomen}
&\uncover<7->{$\displaystyle \begin{aligned}
B_{k,n}(t) &= \frac{1}{(b-a)^n}\binom{n}{k}(t-a)^k(b-t)^{n-k}
\\
B_n(f)(t) &= \sum_{k=0}^n B_{k,n}(t) \cdot f\biggl(\frac{k}{n}\biggr)
\end{aligned}$}
\\
\hline
\end{tabular}
\end{center}
\end{block}}
\end{frame}
