%
% normalbeispiel34.tex
%
% (c) 2021 Prof Dr Andreas Müller, OST Ostschweizer Fachhochschule
%
\bgroup
\definecolor{darkgreen}{rgb}{0,0.6,0}
\definecolor{darkred}{rgb}{0.8,0,0}
\begin{frame}[t]
\frametitle{Beispiele normaler Matrizen für $n=3$}
\vspace{-20pt}
\begin{columns}[t,onlytextwidth]
\begin{column}{0.49\textwidth}
\begin{align*}
A
&=
\begin{pmatrix}
\alpha&\beta &   0  \\
   0  &\alpha&\beta \\
\beta &   0  &\alpha
\end{pmatrix},
\;
A^t=
\begin{pmatrix}
\alpha&   0  &\beta \\
\beta &\alpha&   0  \\
   0  &\beta &\alpha
\end{pmatrix}
&
\uncover<2->{%
&\Rightarrow\left\{
\begin{aligned}
AA^t&=\begin{pmatrix}
\alpha^2+\beta^2 & \alpha\beta      & \alpha\beta      \\
\alpha\beta      & \alpha^2+\beta^2 & \alpha\beta      \\
\alpha\beta      & \alpha\beta      & \alpha^2+\beta^2 
\end{pmatrix}
\\
&\phantom{ooooooooooooooo}\|
\\
A^tA&=\begin{pmatrix}
\alpha^2+\beta^2 & \alpha\beta      & \alpha\beta      \\
\alpha\beta      & \alpha^2+\beta^2 & \alpha\beta      \\
\alpha\beta      & \alpha\beta      & \alpha^2+\beta^2 
\end{pmatrix}
\end{aligned}\right.}
\\
\uncover<3->{
A&=\alpha I + \beta O}\uncover<4->{, O=\begin{pmatrix}0&1&0\\0&0&1\\1&0&0\end{pmatrix}\in \operatorname{O}(3)}
&
\uncover<5->{
&\Rightarrow
\left\{
\begin{aligned}
AA^*&= \alpha^2I^2 + \beta^2
\ifthenelse{\boolean{presentation}}{ \only<6->{I} }{} \only<-5>{OO^*}
+ \alpha\beta(O+O^*)\\
A^*A&= \alpha^2I^2 + \beta^2
\ifthenelse{\boolean{presentation}}{ \only<6->{I} }{} \only<-5>{O^*O}
+ \alpha\beta(O^*+O)
\end{aligned}
\right.}
\\
\uncover<7->{A&=U+V^*,\text{normal}}\uncover<10->{\text{, }
{\color{darkgreen}UV}={\color{darkgreen}VU}}
&
&\uncover<8->{\Rightarrow
\left\{
\begin{aligned}
AA^* &= UU^* + {\color<9->{darkgreen}UV} + {\color<9->{darkred}V^*U^*} + V^*V
\\
A^*A &= U^*U + {\color<9->{darkred}U^*V^*} + {\color<9->{darkgreen}VU} + VV^*
\end{aligned}
\right.}
\end{align*}
\end{column}
\begin{column}{0.49\textwidth}
\end{column}
\end{columns}
\end{frame}
