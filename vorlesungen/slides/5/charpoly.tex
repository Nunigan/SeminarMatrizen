%
% charpoly.tex
%
% (c) 2021 Prof Dr Andreas Müller, Hochschule Rapperswil
%
\begin{frame}[t]
\setlength{\abovedisplayskip}{5pt}
\setlength{\belowdisplayskip}{5pt}
\frametitle{Charakteristisches Polynom über $\mathbb{C}$}
\vspace{-18pt}
\begin{columns}[t,onlytextwidth]
\begin{column}{0.48\textwidth}
\begin{block}{Eigenwerte}
Nur diejenigen $\mu$ kommen in Frage, für die
$A-\mu I$ singulär ist:
\[
\chi_{A}(\mu)
=
\det (A-\mu I) = 0
\]
$\Rightarrow$ $\mu$ ist Nullstelle von $\chi_{A}(X)\in\mathbb{C}[X]$
\end{block}
\uncover<2->{%
\begin{block}{Zerlegung in Linearfaktoren}
$\mu_1,\dots,\mu_n$ die Nullstellen von $\chi_A(X)$:
\[
\chi_A(X)
=
(X-\mu_1)\dots (X-\mu_n)
\]
\end{block}}
\uncover<3->{%
\begin{block}{Fundamentalsatz der Algebra}
Über $\mathbb{C}$ zerfällt jedes Polynom in $\mathbb{C}[X]$ in
Linearfaktoren
\end{block}}
\end{column}
\begin{column}{0.48\textwidth}
\uncover<4->{%
\begin{block}{Minimalpolynom}
Alle Nullstellen von $\chi_A(X)$ müssen in $m_A(X)$ vorkommen
\end{block}}
\uncover<5->{%
\begin{proof}[Beweis]
\begin{enumerate}
\item<6->
$m_A(X) = (X-\lambda) \prod_{i\in I}(X-\mu_i)$
\item<7->
$A-\lambda I$ ist regulär
\end{enumerate}
\uncover<8->{%
\begin{align*}
&\Rightarrow&
m_A(A)&=0
\\
&&
\uncover<9->{
(A-\lambda)^{-1}m_A(A) &=0
}
\\
&&
\uncover<10->{
\prod_{i\in I}(A-\mu_i)&=0,
}
\end{align*}}
\uncover<11->{%
d.~h.~\(
\displaystyle
\overline{m}_A(X)
=
\prod_i{i\in I}(X-\mu_i)
\in
\mathbb{C}[X]
\)}
\end{proof}}
\end{column}
\end{columns}
\end{frame}
