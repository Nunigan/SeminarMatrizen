%
% verzerrung.tex
%
% (c) 2021 Prof Dr Andreas Müller, OST Ostschweizer Fachhochschule
%
\bgroup
\def\r{1.10}
\def\s{1.12}
\def\q{1.23}
\definecolor{darkgreen}{rgb}{0,0.6,0}
\begin{frame}[t]
\frametitle{Verzerrung}
\vspace{-15pt}
\begin{columns}[t,onlytextwidth]
\begin{column}{0.49\textwidth}
\begin{block}{Abbildung $A\colon v\mapsto Av$}
\begin{center}
\begin{tikzpicture}[>=latex,thick,scale=2.5]
\draw[color=blue,line width=1.2pt] (0,0) circle[radius=1];

\coordinate (a1) at (0.974,0.171);
\coordinate (a2) at (0.037,1.018);

\coordinate (v1) at (-0.5216,0.8532);
\coordinate (v2) at (-0.3343,-0.9425);

\foreach \a in {0,5,...,355}{
	\draw[color=red,line width=1.2pt] 
		($cos(\a)*(a1)+sin(\a)*(a2)$) --
		($cos(\a+5)*(a1)+sin(\a+5)*(a2)$);
}
\foreach \a in {1,...,144}{
	\only<\a>{
		\fill[color=red,line width=1.4pt]
			($cos(\a*5)*(a1)+sin(\a*5)*(a2)$) circle[radius=0.03];
		\draw[->,color=red,line width=1.4pt] (0,0) --
			($cos(\a*5)*(a1)+sin(\a*5)*(a2)$);
		\draw[->,color=blue,line width=1.4pt] (0,0) -- ({5*\a}:1);
		\fill[color=blue] ({5*\a}:1) circle[radius=0.03];
		\node[color=blue] at ({5*\a}:\r) {$v$};
		\node[color=red] at ($\s*cos(\a*5)*(a1)+\s*sin(\a*5)*(a2)$)
			{$Av$};
	}
}

\begin{scope}
\clip (-1.2,-1.1) rectangle (1.2,1.1);
\draw[color=darkgreen,line width=0.7pt] ($-2*(v1)$) -- ($2*(v1)$);
\draw[color=darkgreen,line width=0.7pt] ($-2*(v2)$) -- ($2*(v2)$);
\draw[->,color=darkgreen,line width=1.5pt] (0,0) -- (v1);
\draw[->,color=darkgreen,line width=1.5pt] (0,0) -- (v2);
\end{scope}

\draw[->] (-\q,0) -- (1.2,0) coordinate[label={$x$}];
\draw[->] (0,-1.2) -- (0,1.2) coordinate[label={right:$y$}];

\node[color=darkgreen] at (v1) [above left] {$v_1$};
\node[color=darkgreen] at (v2) [below left] {$v_2$};

\end{tikzpicture}
\end{center}
\end{block}
\end{column}
\begin{column}{0.49\textwidth}
\uncover<73->{%
\begin{block}{Abbildung $A\colon v\mapsto (A-\lambda)v$}
\begin{center}
\begin{tikzpicture}[>=latex,thick,scale=2.5]
\draw[color=blue,line width=1.2pt] (0,0) circle[radius=1];

\coordinate (a1) at (0.121,0.343);
\coordinate (a2) at (0.074,0.209);

\coordinate (v1) at (-0.5216,0.8532);
\coordinate (v2) at (-0.3343,-0.9425);

\begin{scope}
\clip (-1.2,-1.2) rectangle (1.2,1.2);
\draw[color=darkgreen,line width=0.7pt] ($-2*(v1)$) -- ($2*(v1)$);
\draw[color=darkgreen,line width=0.7pt] ($-2*(v2)$) -- ($2*(v2)$);
\end{scope}

\foreach \a in {0,5,...,355}{
	\draw[color=red!60,line width=4pt] 
		($cos(\a)*(a1)+sin(\a)*(a2)$) --
		($cos(\a+5)*(a1)+sin(\a+5)*(a2)$);
}
\foreach \a in {73,...,144}{
	\only<\a>{
		\fill[color=red,line width=1.4pt]
			($cos(\a*5)*(a1)+sin(\a*5)*(a2)$) circle[radius=0.03];
		\draw[->,color=red,line width=1.4pt] (0,0) --
			($cos(\a*5)*(a1)+sin(\a*5)*(a2)$);
		\draw[->,color=blue,line width=1.4pt] (0,0) -- ({5*\a}:1);
		\fill[color=blue] ({5*\a}:1) circle[radius=0.03];
		\node[color=blue] at ({5*\a}:\r) {$v$};
		\node[color=red] at ($\s*cos(\a*5)*(a1)+\s*sin(\a*5)*(a2)$)
			{$(A-\lambda)v$};
	}
}

\begin{scope}
\clip (-1.2,-1.1) rectangle (1.2,1.1);
\draw[->,color=darkgreen,line width=1.5pt] (0,0) -- (v1);
\draw[->,color=darkgreen,line width=1.5pt] (0,0) -- (v2);
\end{scope}

\draw[->] (-\q,0) -- (1.2,0) coordinate[label={$x$}];
\draw[->] (0,-1.2) -- (0,1.2) coordinate[label={right:$y$}];

\node[color=darkgreen] at (v1) [above left] {$v_1$};
\node[color=darkgreen] at (v2) [below left] {$v_2$};

\end{tikzpicture}
\end{center}

\end{block}}
\end{column}
\end{columns}
\end{frame}
\egroup
