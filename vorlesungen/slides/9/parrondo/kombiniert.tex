%
% kombiniert.tex -- slide template
%
% (c) 2021 Prof Dr Andreas Müller, OST Ostschweizer Fachhochschule
%
\bgroup
\begin{frame}[t]
\setlength{\abovedisplayskip}{5pt}
\setlength{\belowdisplayskip}{5pt}
\frametitle{Kombiniertes Spiel $C$}
\vspace{-20pt}
\begin{columns}[t,onlytextwidth]
\begin{column}{0.48\textwidth}
\begin{block}{Definition}
Ein fairer Münzwurf entscheidet, ob
Spiel $A$ oder Spiel $B$ gespielt wird
\end{block}
\uncover<2->{%
\begin{block}{Übergangsmatrix}
Münzwurf $X$
\begin{align*}
C
&=
P(X=\text{Kopf})\cdot A
+
P(X=\text{Zahl})\cdot B
\\
&\uncover<3->{=
\begin{pmatrix}
           0&\frac{3}{8}&\frac{5}{8}\\
\frac{3}{10}&          0&\frac{3}{8}\\
\frac{7}{10}&\frac{5}{8}&          0
\end{pmatrix}}
\end{align*}
\end{block}}
\vspace{-8pt}
\uncover<4->{%
\begin{block}{Gewinnerwartung im Einzelspiel}
\[
p=\frac13U
\Rightarrow
U^t(G\odot C)p
\uncover<5->{=
-\frac{1}{30}}
\]
\end{block}}
\end{column}
\begin{column}{0.48\textwidth}
\uncover<6->{%
\begin{block}{Iteriertes Spiel}
\[
\overline{p}=C\overline{p}
\quad
\uncover<7->{\Rightarrow
\quad
\overline{p}=\frac{1}{709}\begin{pmatrix}245\\180\\284\end{pmatrix}}
\]
\end{block}}
\uncover<8->{%
\begin{block}{Gewinnerwartung}
\begin{align*}
E(Z)
&=
U^t (G\odot C) \overline{p}
\uncover<9->{=
\frac{18}{709}}
\end{align*}
\uncover<10->{$C$ ist ein Gewinnspiel!}
\end{block}}
\end{column}
\end{columns}
\end{frame}
\egroup
