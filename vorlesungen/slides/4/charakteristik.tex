%
% charakteristisk.tex
%
% (c) 2021 Prof Dr Andreas Müller, OST Ostschweizer Fachhochschule
%
\begin{frame}[t]
\setlength{\abovedisplayskip}{5pt}
\setlength{\belowdisplayskip}{5pt}
\frametitle{Primkörper und Charakteristik}
\vspace{-20pt}
\begin{columns}[t,onlytextwidth]
\begin{column}{0.48\textwidth}
\begin{block}{Primkörper}
$1\in\Bbbk$
\begin{enumerate}
\item<2->
$n\cdot 1\ne 0\;\forall n\in\mathbb{N}$\uncover<3->{:
$\Rightarrow$
$\mathbb{Z}\subset \Bbbk$}
\uncover<4->{%
$\Rightarrow$
$\mathbb{Q}\subset \Bbbk$}
\item<5->
$\{n\mathbb{Z}\;|\;
\text{$n\cdot 1 = 0$ in $\Bbbk$}\}
=
p\mathbb{Z}$
\uncover<6->{
$\Rightarrow$
$\mathbb{F}_p\subset \Bbbk$}
\end{enumerate}
\end{block}
\uncover<7->{%
\begin{block}{Primkörper}
Der Primkörper $\operatorname{Prim}(\Bbbk)$
eines Körpers $\Bbbk$ ist der kleinste in $\Bbbk$
enthaltene Körper
\end{block}}
\end{column}
\begin{column}{0.48\textwidth}
\uncover<8->{%
\begin{block}{Charakteristik}
\vspace{-10pt}
\[
\operatorname{char}(\Bbbk)
=
\begin{cases}
\uncover<9->{p&\qquad \operatorname{Prim}(\Bbbk) = \mathbb{F}_p}\\
\uncover<10->{0&\qquad \operatorname{Prim}(\Bbbk) = \mathbb{Q}}
\end{cases}
\]
\vspace{-10pt}
\end{block}}
\uncover<11->{%
\begin{block}{Vektorraum}
$\Bbbk$ ist ein Vektorraum über $\operatorname{Prim}(\Bbbk)$
durch Einschränkung der Multiplikation auf $\operatorname{Prim}(\Bbbk)$
(Körperstruktur vergessen)
\end{block}}
\uncover<12->{%
\begin{block}{Endliche Körper}
\begin{itemize}
\item<13->
Endliche Körper haben immer Charakteristik $p\ne 0$
\item<14->
$\Bbbk$ ist eine endlichdimensionaler $\mathbb{F}_p$-Vektorraum
\end{itemize}
\end{block}}
\end{column}
\end{columns}
\end{frame}
