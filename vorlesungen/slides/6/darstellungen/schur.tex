%
% schur.tex -- slide template
%
% (c) 2021 Prof Dr Andreas Müller, OST Ostschweizer Fachhochschule
%
\bgroup
\begin{frame}[t]
\setlength{\abovedisplayskip}{5pt}
\setlength{\belowdisplayskip}{5pt}
\frametitle{Folgerungen aus Schurs Lemma}
\vspace{-20pt}
\begin{columns}[t,onlytextwidth]
\begin{column}{0.48\textwidth}
\begin{block}{Mittelung einer Abbildung}
$h\colon V_1\to V_2$
\[
h^G = \frac{1}{|G|} \sum_{g\in G} \varrho_2(g)^{-1} \circ h \circ \varrho_1(g)
\]
\begin{enumerate}
\item<2-> $\varrho_i$ nicht isomorph $\Rightarrow$ $h^G=0$
\item<3-> $V_1=V_2,\varrho_1=\varrho_2$, $h^G = \frac1n\operatorname{Spur}h$
\end{enumerate}
\end{block}
\uncover<4->{%
\begin{block}{Matrixelemente für $\varrho_i$ nicht isomorph}
$\varrho_i$ nicht isomorph, dann ist
\[
\frac{1}{|G|} \sum_{g\in G} \varrho_1(g^{-1})_{kl}\varrho_2(g)_{uv}=0
\]
für alle $k,l,u,v$
\end{block}}
\end{column}
\begin{column}{0.48\textwidth}
\uncover<5->{%
\begin{block}{Matrixelemente $V_1=V_2$, $\varrho_i$ iso}
Für $k=v$ und $l=u$  gilt
\[
\frac{1}{|G|} \sum_{g\in G} \varrho_1(g^{-1})_{kl} \varrho_2(g)_{uv}
=
\frac1n
\]
und $=0$ sonst
\end{block}}
\end{column}
\end{columns}
\end{frame}
\egroup
