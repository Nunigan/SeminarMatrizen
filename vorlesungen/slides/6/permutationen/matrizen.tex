%
% matrizen.tex -- Darstellung der Permutationen als Matrizen
%
% (c) 2021 Prof Dr Andreas Müller, OST Ostschweizer Fachhochschule
%
\bgroup
\begin{frame}[t]
\setlength{\abovedisplayskip}{5pt}
\setlength{\belowdisplayskip}{5pt}
\frametitle{Permutationsmatrizen}
\vspace{-20pt}
\begin{columns}[t,onlytextwidth]
\begin{column}{0.48\textwidth}
\begin{block}{Permutationsabbildung}
$\sigma\in S_n$ eine Permutation, definiere
\[
f
\colon
e_i \mapsto e_{\sigma(i)}
\]
($e_i$ Standardbasisvektor)
\end{block}
\uncover<2->{%
\begin{block}{Lineare Abbildung}
$f$ kann erweitert werden zu einer linearen Abbildung
\[
\tilde{f}
\colon
\Bbbk^n \to \Bbbk^n
:
\sum_{k=1}^n a_i e_i
\mapsto
\sum_{k=1}^n a_i f(e_i)
\]
\end{block}}
\end{column}
\begin{column}{0.48\textwidth}
\uncover<3->{%
\begin{block}{Permutationsmatrix}
Matrix $A_{\tilde{f}}$ der linearen Abbildung $\tilde{f}$
hat die Matrixelemente
\[
a_{ij}
=
\begin{cases}
1&\qquad i=\sigma(j)\\
0&\qquad\text{sonst}
\end{cases}
\]
\end{block}}
\vspace{-10pt}
\uncover<4->{%
\begin{block}{Beispiel}
\vspace{-10pt}
\[
\begin{pmatrix}
1&2&3&4\\
3&2&4&1
\end{pmatrix}
\mapsto
\begin{pmatrix}
0&0&0&1\\
0&1&0&0\\
1&0&0&0\\
0&0&1&0
\end{pmatrix}
\]
\end{block}}
\vspace{-10pt}
\uncover<5->{%
\begin{block}{Homomorphismus}
Die Abbildung
$S_n\to\operatorname{GL}(\Bbbk)\colon \sigma \mapsto A_{\tilde{f}}$
ist ein Homomorphismus
\end{block}}
\end{column}
\end{columns}
\end{frame}
\egroup
