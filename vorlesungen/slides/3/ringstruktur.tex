%
% ringstruktur.tex
%
% (c) 2021 Prof Dr Andreas Müller, OST Ostschweizer Fachhochschule
%
\begin{frame}[t]
\frametitle{Ringstruktur}
\setlength{\abovedisplayskip}{5pt}
\setlength{\belowdisplayskip}{5pt}
\begin{columns}[t,onlytextwidth]
\begin{column}{0.46\textwidth}
\begin{block}{Ring}
Menge $R$ mit zwei zweistelligen Verknüfpungen $+$ und $\cdot$
mit
\begin{enumerate}
\item<3->
$R$ ist abelsche Gruppe bezüglich $+$
\item<5->
$R\setminus\{0\}$ ist ein Monoid bezüglich $\cdot$
\item<7->
Für alle $a,b,c\in R$
\begin{align*}
a(b+c) &= ab+ac
\\
(a+b)c &= ac+bc
\end{align*}
\end{enumerate}
\end{block}
\end{column}
\begin{column}{0.50\textwidth}
\uncover<2->{%
\begin{block}{Polynomring}
$R$ ein Ring, $R[X]$ ``erbt'' Addition und Multiplikation mit
\begin{enumerate}
\item<4->
$R[X]$ ist abelsche Gruppe bezüglich $+$
\item<6->
$R[X]\setminus\{0\}$ ist ein Monoid bezüglich $\cdot$
\item<8->
Für alle $a,b,c\in R[X]$
\begin{align*}
a(b+c) &= ab+ac
\\
(a+b)c &= ac+bc
\end{align*}
\end{enumerate}
\end{block}}
\end{column}
\end{columns}
\end{frame}
