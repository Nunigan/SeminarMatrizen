%
% ideal.tex
%
% (c) 2021 Prof Dr Andreas Müller, OST Ostschweizer Fachhochschule
%
\begin{frame}[t]
\setlength{\abovedisplayskip}{5pt}
\setlength{\belowdisplayskip}{5pt}
\frametitle{Ideal}
\vspace{-20pt}
\begin{columns}[t,onlytextwidth]
\begin{column}{0.48\textwidth}
\begin{block}{Voraussetzungen}
$R$ ein Ring, $r\in R$
\end{block}
\uncover<2->{%
\begin{block}{Vielfache\uncover<4->{ = Hauptideal}}
Die Menge aller Elemente, die durch $r$ teilbar sind\uncover<3->{:
\[
(r)=rR
\]}
\uncover<4->{heisst {\em Hauptideal}}
\end{block}}
\uncover<5->{%
\begin{block}{Ideal}
$I\subset R$ mit
\(RI\subset I\), \(I+I\subset I\)
\end{block}}
\uncover<6->{%
\begin{block}{Hauptidealring}
Jedes Ideal von $R$ ist ein Hauptideal
\\
\uncover<7->{{\usebeamercolor[fg]{title}Beispiele:}
$\mathbb{Z}$,
$\Bbbk[X]$}
\end{block}}
\end{column}
\begin{column}{0.48\textwidth}
\uncover<8->{%
\begin{block}{Grösster gemeinsamer Teiler}
$a,b\in R$
\begin{align*}
\uncover<9->{(a) + (b)
&= aR + bR}
\intertext{\uncover<10->{ist eine Ideal }\uncover<11->{$\Rightarrow$ ein Hauptideal}}
&\uncover<12->{= cR}\uncover<13->{ = \operatorname{ggT}(a,b)R}
\end{align*}
\uncover<14->{Existenz des $\operatorname{ggT}(a,b)$ ist eine
gemeinsame Eigenschaft}
\end{block}}
\uncover<15->{%
\begin{block}{Allgemein}
\begin{itemize}
\item<16->
Alle euklidischen Ringe sind Hauptidealringe
\item<17->
Alle solchen Ringe verwenden den gleichen Algorithmus
für $\operatorname{ggT}(a,b)$
\end{itemize}
\end{block}}
\end{column}
\end{columns}
\end{frame}
