%
% Quotientenring.tex
%
% (c) 2021 Prof Dr Andreas Müller, OST Ostschweizer Fachhochschule
%
\begin{frame}[t]
\setlength{\abovedisplayskip}{5pt}
\setlength{\belowdisplayskip}{5pt}
\frametitle{Quotientenring}
\vspace{-20pt}
\begin{columns}[t,onlytextwidth]
\begin{column}{0.48\textwidth}
\begin{block}{Quotientenring}
$I\subset R$ ein Ideal
\\
\uncover<2->{
$R/I$ hat eine Ringstruktur:
\begin{align*}
\uncover<3->{\pi(s)&=s+I}
\\
\uncover<4->{\pi(s)\pi(r)&= (s+I)(r+I)}\\
            &\uncover<5->{= sr +\underbrace{sI + rI}_{\subset RI\subset I} + II = sr+I}
\\
\uncover<6->{\pi(s)+\pi(r)&= (s+I)+(r+I)}\\
	     &\uncover<7->{=s+r+I=\pi(s+r)}
\end{align*}}
\end{block}
\vspace{-15pt}
\end{column}
\begin{column}{0.48\textwidth}
\uncover<7->{%
\begin{block}{Beispiele}
\begin{itemize}
\item
$\mathbb{Z}/(n)=\mathbb{Z}/n\mathbb{Z}$,
$\mathbb{F}_p=\mathbb{Z}/(p)=\mathbb{Z}/p\mathbb{Z}$
\item<8->
$p\in\Bbbk[X]$
$\Rightarrow$
$\Bbbk[X]/(p)$ ist ein Ring
\end{itemize}
\end{block}}
\uncover<9->{%
\begin{block}{Algebraideal}
$I\subset A$
\begin{itemize}
\item<10->
$I$ ein Unterraum von $A$ als Vektorraum
\item<11->
$I$ ein Ideal von $A$ als Ring
\end{itemize}
\end{block}}
\uncover<12->{%
\begin{block}{Quotientenalgebra}
$A/I$ ist eine Algebra
\end{block}}
\end{column}
\end{columns}
\end{frame}
