%
% einsetzen.tex
%
% (c) 2021 Prof Dr Andreas Müller, OST Ostschweizer Fachhochschule
%
\begin{frame}[t]
\frametitle{Matrix in ein Polynom einsetzen}
\vspace{-10pt}
\[
\begin{array}{rcrcrcrcrcrcr}
p(X)&=&a_nX^n&+&a_{n-1}X^{n-1}&+&\dots&+&a_2X^2&+&a_1X&+&a_0\phantom{I}\\
\uncover<2->{\bigg\downarrow\hspace*{4pt}}  & &
\uncover<3->{\bigg\downarrow\hspace*{4pt}}  & &
\uncover<4->{\bigg\downarrow\hspace*{10pt}} & &     & &
\uncover<5->{\bigg\downarrow\hspace*{4pt}}  & &
\uncover<6->{\bigg\downarrow\hspace*{2pt}}  & &
\uncover<7->{\bigg\downarrow\hspace*{0pt}}  \\
\uncover<2->{p(A)}&\uncover<3->{=&a_nA^n}&\uncover<4->{+&a_{n-1}A^{n-1}}&\uncover<5->{+&\dots&+&a_2A^2}&\uncover<6->{+&a_1A}&\uncover<7->{+&a_0         I}
\end{array}
\]
\vspace{-10pt}
\begin{columns}[t,onlytextwidth]
\begin{column}{0.48\textwidth}
\uncover<8->{%
\begin{block}{Nilpotente Matrizen}
$p(X) = (X-a)^n$
\[
\uncover<9->{p(A) = 0}
\uncover<10->{
\quad\Rightarrow\quad
\text{$A-aI$ ist nilpotent}}
\]
\end{block}}
\end{column}
\begin{column}{0.48\textwidth}
\uncover<11->{%
\begin{block}{Eigenwerte}
$p(X) = (X-\lambda_1)(X-\lambda_2)$,\\
$A$ eine $2\times 2$-Matrix
\[
\uncover<12->{p(A)=0}
\uncover<13->{\quad\Rightarrow\quad
\left\{
\begin{aligned}
&\text{$A-\lambda_1I$ ist singulär}\\
&\text{$A-\lambda_2I$ ist singulär}
\end{aligned}
\right.}
\]
\end{block}}
\end{column}
\end{columns}

\end{frame}
