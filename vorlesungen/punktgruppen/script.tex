\documentclass[a4paper]{article}

\usepackage{amsmath}
\usepackage{amssymb}

\usepackage[cm]{manuscript}
\usepackage{xcolor}

\newcommand{\scene}[1]{\par\noindent[ #1 ]\par}
\newenvironment{totranslate}{\color{blue!70!black}}{}

\begin{document}
\section{Das sind wir}
\scene{Camera}

\section{Ablauf}
Zuerst werden wir Symmetrien in 2 Dimensionen anschauen, dann \"uberlegen wir
kurz was es heisst f\"ur eine Symmetrie ``algebraisch'' zu sein. Von da aus
kommt die dritte Dimension hinzu, die man besser mit Matrizen verstehen kann.
Mit der aufgebauten Theorie werden wir versuchen Kristalle zu klassifizieren.
Und zum Schluss kommen wir zu Anwendungen, welche f\"ur Ingenieure von
Interesse sind.

\section{intro}
\scene{Spontan}

\section{2D Geometrie}
\scene{Intro}
Wir fangen mit den 2 dimensionalen Symmetrien an, da man sie sich am
einfachsten vorstellen kann. Eine Symmetrie eines Objektes beschreibt eine
Aktion, welche nachdem sie auf das Objekt wirkt, das Objekt wieder gleich
aussehen l\"asst.

\scene{Viereck}
Die einfachste Aktion, ist das Viereck zu nehmen, und wieder hinzulegen.
Eine andere Aktion k\"onnte sein, das Objekt um eine Achse zu spiegeln,
oder eine Rotation um 90 Grad.

\scene{Zyklische Gruppe}
Fokussieren wir uns auf die einfachste Klassen von Symmetrien: diejenigen die
von einer reinen Drehung generiert werden. Wir sammeln diese in einer Gruppe
\(G\), und notieren das sie von eine Rotation \(r\) generiert worden sind, mit
diesen spitzen Klammern.

Nehmen wir als Beispiel dieses Pentagon. Wenn wir \(r\) 5-mal anwenden, ist es
dasselbe als wenn wir nichts gemacht h\"atten. Wenn wir es noch ein 6. mal
drehen, entspricht dies dasselbe wie \(r\) nur 1 mal zu nutzen.

\scene{Notation}
So, die Gruppe setzt sich zusammen aus dem neutralen Element, und den Potenzen
1 bis 4 von \(r\). Oder im allgemein Gruppen mit dieser Struktur, in welcher die
Aktion \(n-1\) mal angewendet werden kann, heissen ``Zyklische Gruppe''.

\scene{Diedergruppe}
Nehmen wir nun auch noch die Spiegeloperation \(\sigma\) dazu. Weil wir jetzt 2
Operationen haben, m\"ussen wir auch im Generator schreiben wie sie
zusammenh\"angen.  Schauen wir dann uns genauer diesen Ausdr\"uck an.  Zweimal
Spielegeln ist \"aquivalent zum neutralen Element, sowie 4 mal um 90 Grad
drehen und 2 Drehspiegelungen, welche man auch Inversion nennt.

\scene{Notation}
Daraus k\"onnen wir wieder die ganze Gruppe erzeugen, die im allgemeinen den
Symmetrien eines \(n\)-gons entsprechen.

\scene{Kreisgruppe}
Bis jetzt hatten wir nur diskrete Symmetrien, was nicht zwingend der Fall sein
muss. Ein Ring kann man kontinuierlich drehen, und sieht dabei immer gleich
aus.

Diese Symmetrie ist auch als Kreisgruppe bekannt, die man sch\"on mit dem
komplexen Einheitskreis definieren kann.

\section{Algebra}
\scene{Produkt mit \(i\)}
\"Uberlegen wir uns eine spezielle algebraische Operation: Multiplikation mit
der imagin\"aren Einheit. \(1\) mal \(i\) ist gleich \(i\). Wieder mal \(i\)
ist \(-1\), dann \(-i\) und schliesslich kommen wir z\"uruck auf \(1\).  Diese
fassen wir in eine Gruppe \(G\) zusammen. Oder sch\"oner geschrieben:. Sieht das
bekannt aus?

\scene{Morphismen}
Das Gefühl, dass es sich um dasselbe handelt, kann wie folgt formalisiert
werden.  Sei \(\phi\) eine Funktion von \(C_4\) zu \(G\) und ordnen wir zu
jeder Symmetrieoperation ein Element aus \(G\). Wenn man die Zuordnung richtig
definiert, dann sieht man die folgende Eigenschaft: Eine Operation nach eine
andere zu nutzen, und dann die Funktion des Resultats zu nehmen, ist gleich wie
die Funktion der einzelnen Operazionen zu nehmen und die Resultate zu
multiplizieren. Dieses Ergebnis ist so bemerkenswert, dass es in der Mathematik
einen Namen bekommen hat: Homorphismus, von griechisch "homos" dasselbe und
"morphe" Form.  Manchmal auch so geschrieben. Ausserdem, wenn \(\phi\) eins zu
eins ist, heisst es \emph{Iso}morphismus: "iso" gleiche Form. Was man
typischerweise mit diesem Symbol schreibt.

\scene{Animation}
Sie haben wahrscheinlich schon gesehen, worauf das hinausläuft.  Dass die
zyklische Gruppe \(C_4\) und \(G\) isomorph sind ist nicht nur Fachjargon der
mathematik, sondern sie haben wirklich die selbe Struktur.

\scene{Modulo}
Das Beispiel mit der komplexen Einheit, war wahrscheinlich nicht so
\"uberraschend. Aber was merkw\"urdig ist, ist das Beziehungen zwischen
Symmetrien und Algebra auch in Bereichen gefunden werden, welche auf den ersten
Blick, nicht geomerisch erscheinen. Ein R\"atsel für die Neugierigen: die Summe
in der Modulo-Arithmetik.  Als Hinweis: Um die Geometrie zu finden denken Sie
an einer Uhr.

\section{3D Geometrie}
2 Dimensionen sind einfacher zu zeichnen, aber leider leben wir im 3
dimensionalen Raum.

\scene{Zyklische Gruppe}
Wenn wir unser bekanntes Viereck mit seiner zyklischer Symmetrie in 3
Dimensionen betrachten, k\"onnen wir seine Drehachse sehen.

\scene{Diedergruppe}
Um auch noch die andere Symmetrie des Rechteckes zu sehen, ben\"otigen wir eine
Spiegelachse \(\sigma\), die hier eine Spiegelebene ist.

\scene{Transition}
Um die Punktsymmetrien zu klassifizieren orientiert man sich an einer Achse, um
welche sich die meisten Symmetrien drehen. Das geht aber nicht immer, wie beim
Tetraeder.

\scene{Tetraedergruppe}
Diese Geometrie hat 4 gleichwertige Symmetrieachsen, die eben eine
Symmetriegruppe aufbauen, welche kreativer weise Tetraedergruppe genannt wird.
Vielleicht fallen Ihnnen weitere Polygone ein mit dieser Eigenschaft, bevor wir
zum n\"achsten Thema weitergehen.

\section{Matrizen}
\scene{Titelseite}
Nun gehen wir kurz auf den Thema unseres Seminars ein: Matrizen.  Das man mit
Matrizen Dinge darstellen kann, ist keine Neuigkeit mehr, nach einem
Semester MatheSeminar.  Also überrascht es wohl auch keinen, das man alle
punktsymmetrischen Operationen auch mit Matrizen Formulieren kann.

\scene{Matrizen}

Sei dann \(G\) unsere Symmetrie Gruppe, die unsere abstrakte Drehungen und
Spiegelungen enth\"ahlt. Die Matrix Darstellung dieser Gruppe, ist eine
Funktion gross \(\Phi\), von \(G\) zur orthogonalen Gruppe \(O(3)\), die zu
jeder Symmetrie Operation klein \(g\) eine Matrix gross \(\Phi_g\) zuordnet.

Zur Erinnerung, die Orthogonale Gruppe ist definiert als die Matrizen, deren
transponierte auch die inverse ist. Da diese Volumen und Distanzen erhalten,
natuerlich nur bis zu einer Vorzeichenumkehrung, macht es Sinn, dass diese
Punksymmetrien genau beschreiben.

Nehmen wir die folgende Operationen als Beispiele. Die Matrix der trivialen
Operation, dass heisst nichts zu machen, ist die Einheitsmatrix. Eine
Spiegelung ist dasselbe aber mit einem Minus, und Drehungen sind uns schon
dank Herrn M\"uller bekannt.

\section{Kristalle}
\scene{Spontan}

\section{Piezo}
\scene{Spontan}

\section{Licht}
Als Finale, haben wir ein schwieriges Problem aus der Physik. Das Ziel dieser
Folie ist nicht jedes Zeichen zu versehen, sondern zu zeigen wie man von hier
weiter gehen kann. Wir mochten sehen wie sich Licht in einem Kristall verhaltet.
Genauer, wir m\"ochten die Amplitude einer
elektromagnetischer Welle in einem Kristall beschreiben.

Das Beispiel richtet sich mehr an Elektrotechnik Studenten, aber die Theorie
ist die gleiche bei mechanischen Wellen in Materialien mit einer
Spannungstensor wie dem, den wir letzte Woche gesehen haben.
% Ganz grob gesagt, ersetzt man E durch Xi und epsilon durch das Sigma.

Um eine Welle zu beschreiben, verwenden wir die Helmholtz-Gleichung, die einige
von uns bereits in anderen Kursen gel\"ost haben.  Schwierig wird aber dieses
Problem, wenn der Term vor der Zeitableitung ein Tensor ist (f\"ur uns eine Matrix).

Zur Vereinfachung werden wir eine ebene Welle verwenden. Setzt man dieses E in
die Helmholtz-Gleichung ein, erhält man folgendes zurück: ein Eigenwertproblem.

Physikalisch bedeutet dies, dass die Welle in diesem Material ihre Amplitude in
Abhängigkeit von der Ausbreitungsrichtung ändert.  Und die Eigenwerte sagen
aus, wie stark die Amplitude der Welle in jeder Richtung skaliert wird.

Ich sagte, in jede Richtung skaliert, aber welche Richtungen genau?
Physikalisch hängt das von der kristallinen Struktur des Materials ab, aber
mathematisch können wir sagen: in Richtung der Eigenvektoren!  Aber diesen
Eigenraum zu finden, in dem die Eigenvektoren wohnen, ist beliebig schwierig. 

Hier kommt unsere Gruppentheorie zu Hilfe. Wir können die Symmetrien unseres
Kristalls zur Hilfe nehmen. Zu jeder dieser Symmetrien lässt sich bekanntlich eine
einfache Matrix finden, deren Eigenraum ebenfalls relativ leicht zu finden ist.
Zum Beispiel ist der Eigenraum der Rotation \(r\), die Rotationsachse, für die
Reflexion \(\sigma\) eine Ebene, und so weiter.

Nun ist die Frage, ob man diese Eingenraume der Symmetrienoperationen
kombinieren kann um den Eigenraum des physikalisches Problems zu finden.

Aber leider ist meine Zeit abgelaufen in der Recherche, also müssen Sie mir 2
Dingen einfach glauben, erstens dass es einen Weg gibt, und zweitens dass eher
nicht so schlimm ist, wenn man die Notation einmal gelernt hat.

Nachdem wir an, wir haben den Eigenraum U gefunden, dann können wir einen
(Eigen)Vektor E daraus nehmen und in ihm direkt lambda ablesen. Das sagt uns,
wie die Amplitude der Welle, in diese Richtung gedämpft wurde.

Diese Methode ist nicht spezifisch für dieses Problem, im Gegenteil, ich habe
gesehen, dass sie in vielen Bereichen eingesetzt wird, wie z.B.:
Kristallographie, Festkörperphysik, Molekülschwingungen in der Quantenchemie
und numerische Simulationen von Membranen.

\section{Outro}
\scene{Camera}

\end{document}
% vim:et ts=2 sw=2:
