%
% slides.tex -- XXX
%
% (c) 2017 Prof Dr Andreas Müller, Hochschule Rapperswil
%
\section{Graphen}
\folie{8/graph.tex}
\folie{8/dgraph.tex}

\section{Matrizen}
\folie{8/grad.tex}
\folie{8/inzidenz.tex}
\folie{8/inzidenzd.tex}

\section{Pfade}
\folie{8/pfade/adjazenz.tex}
\folie{8/pfade/langepfade.tex}
\folie{8/pfade/beispiel.tex}
\folie{8/pfade/gf.tex}

\ifthenelse{\boolean{presentation}}{
\section{Floyd-Warshall}
\folie{8/floyd-warshall/problem.tex}
\folie{8/floyd-warshall/rekursion.tex}
\folie{8/floyd-warshall/iteration.tex}
\folie{8/floyd-warshall/wege.tex}
\folie{8/floyd-warshall/wegiteration.tex}
}{}

\section{Flüsse}
\folie{8/diffusion.tex}
\ifthenelse{\boolean{presentation}}{
\folie{8/tokyo/google.tex}
\folie{8/tokyo/bahn0.tex}
\folie{8/tokyo/bahn1.tex}
\folie{8/tokyo/bahn2.tex}
}{}

\folie{8/laplace.tex}
\folie{8/produkt.tex}
\folie{8/fourier.tex}
\folie{8/spanningtree.tex}

\section{Markov-Ketten}
\folie{8/markov/google.tex}
\folie{8/markov/markov.tex}
\folie{8/markov/stationaer.tex}
\folie{8/markov/irreduzibel.tex}
\folie{8/markov/pf.tex}

